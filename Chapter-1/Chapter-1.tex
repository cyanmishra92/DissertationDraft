\chapter{Introduction}

%In today's age of data driven computing, mobile devices and sensor networks are playing a major role. 
Innovations in low-power computing, artificial intelligence, and communication technologies have given rise to the generation of intelligent connected devices which constitute the Internet of Things (IoT). While IoT platforms are diverse, mobility is common and wearables are an increasingly important class of platforms:
%While, IoT spans from the connected home appliances to connected vehicles, wearable devices and hand-held mobile devices form a major portion of IoT. 
For instance, the number of connected wearable devices is expected to increase to 1,105 million in 2022~\cite{Cisco-report}. At the same time, application demands for these devices are also growing: IoT devices, including wearables, are now expected to participate in producing rapid inferences as part of the increasingly complex tasks enabled by machine learning algorithms~\cite{netadapt}.
Furthermore, considering the role of wearable devices, especially in wireless body area networks, emerging applications and research are now targeted towards optimizing the wearable devices, making them more user friendly and more intelligent. These devices are now expected to be capable of handling many complex tasks, such as human activity recognition (HAR), machine translation, biometric authentication, fall and accident detection. % ECG measurement etc. 

 %~\cite{eyeris}~\cite{deepcompression}~\cite{DRQ}. 
However, given their form-factor constraints, it is challenging to execute compute-intensive inference tasks directly on wearable platforms due to their limited energy storage and low-power operation. %excepting a few inherently infrequent small-tasks, like biometric authentication. 
Therefore, complex inference tasks, such as human activity recognition (HAR) %~\cite{HAR-fixmeCITEFORHAR}, 
are typically offloaded either to the cloud or a connected (to internet) host device, such as a smartphone. 
%Offloading 
%occur even though it is more energy-efficient to process the data on the device itself~\cite{ResiRCA},
%is required due to the limited compute and energy capacities of the wearable devices. 
To alleviate energy provisioning challenges, recent works~\cite{IntBeyondEdge, chinchilla} proposed energy harvesting, along with multiple compiler and runtime optimizations, to increase local compute at the edge. Furthermore, non-volatile processors (NVP)~\cite{NVPMa} have been proposed to maximize the forward progress when utilizing harvested energy sources.  

Recent works~\cite{spendthrift,ResiRCA} suggest that processing data at the source is more efficient than sending them to the cloud to get back the results, because of the power and latency overhead resulting from data communication. Hence, many of them propose optimizations to efficiently execute DNNs on low powered IoT devices~\cite{IntBeyondEdge,chinchilla}. Other works~\cite{NVPMa,spendthrift,IntBeyondEdge,chinchilla} have proposed using energy harvesting (EH) as a solution to provide additional energy and increase battery life of the IoT devices. These works propose software, hardware and compiler level optimizations to build a battery-less system powered by harvested energy only, and hence making the sensing devices capable of performing inference tasks. Furthermore, EH, in addition to prolonging the device lifetime, can help us reduce the environmental impact of batteries~\cite{incidental}. However, energy harvesting is no panacea for the fickle nature of harvested energy, and, to tackle this, recent works have proposed nonvolatile processors (NVP)~\cite{NVPMa,ResiRCA} to ensure sufficient forward progress.  

NVP combined with other optimization techniques can enable EH based sensing devices to perform complex computations, there by reducing the burden from the host mobile device. However, these sensor-node-level optimizations are neither adequate nor suitable for a sensor network where multiple sensors (with possibly different sensing modalities) collectively work towards achieving a single goal. Such sensor networks are very common, and wireless body area network is one of the most prevalent example where multiple wearable sensing devices, such as, smart watches, smart shoes, smart chest bands, etc., work together for health monitoring, or activity recognition. Although, fusing sensor data is not uncommon, and can be a possible solution in these cases, such data fusion would require a central point where the inference could take place and hence requiring the EH-sensors to perform a power hungry data communication. Furthermore, the intermittent failure of one or more participating sensors would result in many partially completed computation and hence synchronizing all of them together extremely difficult. Evidently, in an energy harvesting wireless sensor network (EH-WSN), communicating all the data to a single host is infeasible and so is expecting all the computations (inferences) to be completed in the harvested energy budget, and there by introducing several challenges in performing inference on an EH-WSN. Things become even more complicated when we try to introduce learning in such scenarios. Sensor networks, deployed in real environments, continuously gather data and encounter previously unseen scenarios along with drifts in the already learnt tasks. Therefore, energy efficient, quick and robust learning with fewer data samples at the edge might become of extreme importance, since these small devices can nether afford to store a large set of data and learn in a traditional manner, or send huge volumes of data to another device (like a host or cloud) for a remote learning thanks to the intermittent energy availability and limited compute and storage.  


The objective of this proposal is to investigate various challenges in deploying "\textit{intermittent}" (due to EH or any other reason) yet "\textit{Intelligent}" wireless sensor network, where not only the individual sensors, but also entire network works intelligently as a full system, even during extreme resource scarcity. First, we look into the challenges and opportunities in deploying a real-time, complex yet accurate inference in a sensor network (more precisely a wireless body area network). While optimized DNNs and energy efficient hardware do maximize the compute, it is more important know the task distribution as well as \textit{scheduling} to make sure that the work is distributed as per the sensors' capabilities, power budget and task at hand. Furthermore, we try to look into \textit{ensemble learning} for making these sensors robust while working in cohesion as well as catering to peculiarities of the individual users.

Second, we observe the fraction of inference that could be completed at the energy harvesting edge with all the DNN optimizations paired with efficient scheduling, and transfer the rest of the compute to a powered host. However, given the heavy power and latency demand of the communication protocols, we look into efficient, feature aware yet application agnostic way of compressing the data, so that the offload could be both latency and energy efficient. We try to leverage \textit{coresets} for this purpose, and further optimize them with quantization and \textit{application awareness} to make the data transfer seamless. Finally, after tackling few of the inference issues, we try to find challenges and opportunities in enabling learning in these networks. We start by looking into the breakdown of compute and energy requirement for performing in-situ learning and look for opportunities in bringing in the state-of-the art learning mechanism to cater towards the intermittent and energy constraint nature of the sensor networks.  

%(due to .....)

\section{Proposal Organization}

This proposal document is organized as follows : In Chapter 2, we discuss the related work aimed at addressing different solutions towards deploying complex compute in EH-WSNs. Chapter 3 discusses intelligent scheduling scheme for maximising compute at the sensor nodes. Chapter 4 discusses efficient communication between the sensor network and a host. In Chapter 5, we discuss opportunity for efficient training in the EH-WSNs. Chapter 6 concludes this proposal.