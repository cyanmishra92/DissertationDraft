Enabling DNN inference on edge devices has been gaining recent traction, especially for tasks like HAR. However, the compute heavy DNNs make it challenging because of their power requirements, especially in EH-WSNs. %, owing to the fickle nature of EH and the minimal power availability. 
Our proposal, \emph{Origin}, holistically looks into multiple aspects of deploying a DNN on an EH-WSN for the purpose of HAR. \emph{Origin} combines an intelligent activity aware scheduler with an adaptive and light weight ensemble learning method. % to bridge the accuracy gap between an EH-WSN running a DNN with many failures and a fully powered system running the same. 
Our experiments shows that DNN inference using \emph{Origin}, running on a harvested energy only system, is more accurate than energy-constraint-optimized DNNs, running on a fully-powered system. Although the current work is limited to HAR, this can further be extended to many suitable tasks which need to leverage a distributed sensor system for DNN inference. We believe that the co-optimization of deep learning and energy harvesting techniques for edge devices will further invigorate research on the next generations of intelligent and sustainable IoT platforms.     