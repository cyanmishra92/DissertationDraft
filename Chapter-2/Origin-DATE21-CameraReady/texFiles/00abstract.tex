There is an increasing demand for performing machine learning tasks, such as human activity recognition (HAR) on emerging ultra-low-power internet of things (IoT) platforms. Recent works show substantial efficiency boosts from performing inference tasks directly on the IoT nodes rather than merely transmitting raw sensor data. However, the computation and power demands of deep neural network (DNN) based inference pose significant challenges when executed on the nodes of an energy-harvesting wireless sensor network (EH-WSN). Moreover, managing inferences requiring responses from multiple energy-harvesting nodes imposes challenges at the system level in addition to the constraints at each node.  

This paper presents a novel scheduling policy along with an adaptive ensemble learner to efficiently perform HAR on a distributed energy-harvesting body area network. Our proposed policy, \textit{Origin}, strategically ensures efficient and accurate individual inference execution at each sensor node by using a novel activity-aware scheduling approach. It also leverages the continuous nature of human activity when coordinating and aggregating results from all the sensor nodes to improve final classification accuracy. Further, \textit{Origin} proposes an adaptive ensemble learner to personalize the optimizations based on each individual user. Experimental results using two different HAR data-sets show \textit{Origin}, while running on harvested energy, to be at least 2.5\% more accurate than a classical battery-powered energy aware HAR classifier continuously operating at the same average power. 