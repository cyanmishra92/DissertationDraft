There is an increasing demand for intelligent processing on ultra-low-power internet of things (IoT) device. Recent works have shown substantial efficiency boosts by executing inferences directly on the IoT device (node) rather than transmitting data. However, the computation and power demands of Deep Neural Network (DNN)-based inference pose significant challenges in an energy-harvesting wireless sensor network (EH-WSN). Moreover, these tasks often require responses from multiple physically distributed EH sensor nodes, which impose crucial system optimization challenges in addition to per-node constraints. To address these challenges, we propose \emph{Seeker}, a hardware-software co-design approach for increasing on-sensor computation, reducing communication volume, and maximizing inference completion, without violating the quality of service, in EH-WSNs coordinated by a mobile device. \emph{Seeker} uses a \emph{store-and-execute} approach to complete a subset of inferences on the EH sensor node, reducing communication with the mobile host. Further, for those inferences unfinished because of the  harvested energy constraints, it leverages task-aware coreset construction to efficiently communicate compact features to the host device. We evaluate  \emph{Seeker} for human activity recognition, as well as predictive maintenance and show $\approx 8.9\times$ reduction in communication data volume with $86.8\%$ accuracy, surpassing the $81.2\%$ accuracy of the state-of-the-art.

%%%%

%{\bf KEEP ALL DECIMAL POINTS TO ONE DIGIT}
%\textcolor{red}{(one line for result)}