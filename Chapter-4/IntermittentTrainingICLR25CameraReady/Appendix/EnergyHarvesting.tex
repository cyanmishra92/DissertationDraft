A typical energy harvesting (EH) setup captures and converts environmental energy into usable electrical power, which can then support various electronic devices. Here's a simplified breakdown of the process:

\begin{enumerate}
    \item \textbf{Energy Capture}: The setup begins with a harvester, such as a solar panel, piezoelectric sensor, or thermocouple. These devices are designed to collect energy from their surroundings—light, mechanical vibrations, or heat, respectively.
    \item \textbf{Power Conditioning}: Once energy is harvested, it often needs to be converted and stabilized for use. This is done using a rectifier, which transforms alternating current (AC) into a more usable direct current (DC).
    \item \textbf{Voltage Regulation}: After rectification, the power might not be at the right voltage for the device it needs to support. A matching circuit, including components like buck or boost converters, adjusts the voltage to the appropriate level, ensuring the device receives the correct current and voltage.
    \item \textbf{Energy Storage}: Finally, to ensure a continuous power supply even when the immediate energy source is inconsistent (like when a cloud passes over a solar panel), the system includes a temporary storage unit, such as a super-capacitor. This component helps smooth out the supply, providing steady power to the compute circuit.
\end{enumerate}

By integrating these components, an EH system can sustainably power devices without relying on traditional power grids, making it ideal for remote or mobile applications.

%\sec
