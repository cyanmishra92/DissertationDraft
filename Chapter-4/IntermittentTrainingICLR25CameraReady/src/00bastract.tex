%%%%%

% The deployment of Deep Neural Networks (DNNs) in energy-constrained environments, such as Energy Harvesting Wireless Sensor Networks (EH-WSNs), presents unique challenges, primarily due to the ``intermittent'' nature of power availability. To address these challenges, this study introduces and evaluates a novel training methodology tailored for DNNs operating within such contexts. In particular, we propose a dynamic dropout technique that adapts to both the architecture of the device and the variability in energy availability inherent in energy harvesting scenarios. 

% Our proposed approach leverages a device model that incorporates specific parameters of the network architecture and the energy harvesting profile to optimize dropout rates dynamically during the training phase. By modulating the network’s training process based on predicted energy availability, our method not only conserves energy but also ensures sustained learning and inference capabilities under power constraints. Our preliminary results demonstrate that this strategy provides $6\%$ -- $22\%$ accuracy improvements compared to the state of the art with $\le 5\%$ additional compute. This paper details the development of the device model, describes the integration of energy profiles with intermittency aware dropout and quantization algorithms, and presents a comprehensive evaluation  of the proposed approach using real-world energy harvesting data. The work also includes a new dataset towards deploying energy harvesting based computation in real world.
\begin{abstract}
The deployment of Deep Neural Networks (DNNs) in energy-constrained environments, such as Energy Harvesting Wireless Sensor Networks (EH-WSNs), introduces significant challenges due to the intermittent nature of power availability. This study introduces \textit{NExUME}, a novel training methodology designed specifically for DNNs operating under such constraints. We propose a dynamic adjustment of training parameters---dropout rates and quantization levels---that adapt in real-time to the available energy, which varies in energy harvesting scenarios. 

This approach utilizes a model that integrates the characteristics of the network architecture and the specific energy harvesting profile. It dynamically adjusts training strategies, such as the intensity and timing of dropout and quantization, based on predictions of energy availability. This method not only conserves energy but also enhances the network’s adaptability, ensuring robust learning and inference capabilities even under stringent power constraints. Our results show a 6\% to 22\% improvement in accuracy over current methods, with an increase of less than 5\% in computational overhead. This paper details the development of the adaptive training framework, describes the integration of energy profiles with dropout and quantization adjustments, and presents a comprehensive evaluation using real-world data. Additionally, we introduce a novel dataset aimed at furthering the application of energy harvesting in computational settings.
\end{abstract}

%%%%%

