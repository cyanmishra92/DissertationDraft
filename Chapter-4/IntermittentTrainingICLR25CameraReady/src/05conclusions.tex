%%%%%
% This paper introduces and evaluates NExUME (Neural Execution Under Intermittent Environment), a versatile framework enhancing DNN operations in energy-harvesting, intermittently-powered settings like remote sensing and wearable technologies. It dynamically integrates neural architecture search, dropout, and quantization adapted to energy variability, thereby significantly advancing the efficiency of DNN deployments. Key innovations include DynFit for real-time adaptive dropout and quantization, and DynInfer for responsive task scheduling. Further, another component, DynAgent,  monitors and adapts to the shifting energy and hardware landscape. These contributions notably elevate IoT sustainability, promote accessibility in power-unstable regions, and catalyze advancements in energy harvesting and embedded ML technologies. NExUME sets a new standard for future research and deployment of energy-efficient, robust neural networks in diverse application scenarios.

This study presents NExUME, an advanced framework designed to optimize the training and inference phases of deep neural networks within the constraints of intermittently powered, energy-harvesting devices. By integrating adaptive neural architecture and energy-aware training techniques, NExUME significantly enhances the viability of deploying machine learning models in environments with limited and unreliable energy sources. The results from our extensive evaluations demonstrate that NExUME can substantially outperform traditional methods in energy-constrained settings, with improvements in accuracy and efficiency that facilitate real-world applications in remote and wearable technology. Specifically, improvements ranging from 6.10\% to 17.13\% over existing methods highlight NExUME's capability to adapt dynamically to fluctuating energy conditions, ensuring both operational longevity and computational integrity. The broader implication of this work extends beyond technological advancements, suggesting a paradigm shift in how the machine learning community approaches the design and deployment of systems in energy-limited environments. By prioritizing energy efficiency and system adaptability, NExUME contributes to the sustainability and accessibility of machine learning solutions, enabling their deployment in regions where power infrastructure is absent or unreliable. This is particularly crucial in developing regions where such technology can drive innovation in healthcare, agriculture, and education. Furthermore, the development of energy-efficient, adaptive systems like NExUME is aligned with the growing need for sustainable computing practices across all disciplines of technology. It challenges the machine learning community to consider not only the accuracy and efficiency of algorithms but also their environmental impact and accessibility, ensuring a broader positive social impact.

%In conclusion, NExUME sets a new benchmark for energy-efficient computing in intermittently powered systems, marking a significant step forward in the practical deployment of machine learning technologies in constrained environments. Future work will focus on extending these methodologies to a wider array of network architectures and applications, potentially transforming how computational tools are used in energy-constrained settings worldwide.

%%%%%


