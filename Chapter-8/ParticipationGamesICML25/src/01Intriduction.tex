\section{Introduction}
The rapid proliferation of the Internet of Things (IoT) has sparked a tremendous growth in the scale and diversity of sensor deployments, from smart homes to expansive industrial and environmental monitoring systems. As these networks continue to expand, sustaining continuous operation in the face of finite power sources becomes a paramount concern. To address this, energy harvesting (EH) technologies have emerged as a viable solution, enabling sensors to convert ambient energy (e.g., solar, thermal, or vibration) into electrical power. This approach promises perpetual, maintenance-free operation, significantly reducing environmental impact and long-term operational costs.

Despite these advantages, large-scale EH Wireless Sensor Networks (EH-WSNs) remain inherently uncertain. Ambient energy availability varies over time and space, leading to fluctuating sensor activity levels and intermittent participation in both training and inference tasks. Some sensors may frequently become inactive or produce low-quality data due to energy scarcity or environmental noise. Consequently, the mere presence of numerous EH sensors does not guarantee robust and reliable performance for complex tasks such as image recognition, acoustic surveillance, or precision agriculture monitoring.

Achieving accurate inference in these complex scenarios depends on effective coordination. Multiple sensors observing the same phenomenon from different angles can collectively provide more comprehensive and reliable insights than any single sensor could. However, requiring all sensors to participate at all times is impractical, as it drains energy reserves too quickly. Conversely, simplistic policies—such as selecting only the highest-energy sensors—ignore factors like data relevance, sensor quality, and the strategic implications of current participation on future network states.

This challenge motivates the need for intelligent, context-aware participation strategies that dynamically determine which sensors should engage during both the \emph{training} phase—where global model parameters are periodically fine-tuned or updated—and the \emph{inference} phase—where newly observed data are aggregated to produce predictions. Sensors must carefully balance immediate accuracy gains against conserving energy for future tasks, while also anticipating the behavior of other sensors that may be collaborating or competing.

To address these interdependent decisions, we employ a game-theoretic framework. Unlike simple heuristic methods that ignore future resource allocation or complex approaches like reinforcement learning that may be too costly to implement, game theory provides equilibrium guarantees. By modeling each sensor as a rational player aiming to optimize its own long-term utility, we achieve stable, cooperative equilibria where no sensor can improve its outcome through unilateral deviation. This strategic equilibrium underpins both training and inference participation decisions, ensuring that the sensors most likely to improve the global model—given their energy, data quality, and network conditions—are the ones that engage.

To refine the global model parameters without incurring continuous on-edge training costs, we adopt a federated learning paradigm adapted to EH-WSNs. Rather than relying on persistent, centralized updates or continuous federated aggregation, we perform \emph{periodic} or \emph{equilibrium-driven} fine-tuning rounds. These updates occur only when sensors have sufficient energy to participate meaningfully, guided by the game-theoretic equilibrium strategy. By integrating the game-theoretic approach with federated learning principles, we reduce communication overhead and ensure that contributions to model updates come from sensors best positioned to improve accuracy under energy constraints and uncertain availability. Our key contributions are as follows:

% \begin{itemize}
%     \item \textbf{Game-Theoretic Participation Strategy:} We develop a novel game-theoretic model for EH-WSNs that applies to both training and inference phases. This model balances anticipated energy availability, local data quality, and global benefit to establish stable and cooperative equilibria, optimizing the energy-accuracy trade-offs.
    
%     \item \textbf{Federated Learning Integration:} We introduce a federated learning-based framework tailored for intermittent participation and heterogeneous data quality. Unlike continuous on-edge training, we employ periodic or triggered fine-tuning sessions aligned with equilibrium strategies, ensuring robust and progressively improving global models.
    
%     \item \textbf{Joint Optimization of Training and Inference:} Our unified solution aligns training participation decisions with inference needs. Sensors strategically decide when to expend energy on local model updates and when to engage in inference tasks, ultimately maximizing their long-term contribution to the network’s performance.
    
%     \item \textbf{Demonstrated Performance Gains:} Through simulations \textcolor{red}{(add simulation details later)}, we show that our integrated framework outperforms baseline approaches—such as always-on participation or simplistic energy-based selection—by achieving higher inference accuracy, lower energy consumption, and more sustainable long-term operation in EH-WSNs.
% \end{itemize}

\noindent$\bullet$ \textbf{Game-Theoretic Participation Strategy:} We develop a novel game-theoretic model for EH-WSNs that applies to both training and inference phases. This model balances anticipated energy availability, local data quality, and global benefit to establish stable and cooperative equilibria, optimizing the energy-accuracy trade-offs.
    
\noindent$\bullet$ \textbf{Federated Learning Integration:} We introduce a federated learning-based framework tailored for intermittent participation and heterogeneous data quality. Unlike continuous on-edge training, we employ periodic or triggered fine-tuning sessions aligned with equilibrium strategies, ensuring robust and progressively improving global models.
    
\noindent$\bullet$ \textbf{Joint Optimization of Training and Inference:} Our unified solution aligns training participation decisions with inference needs. Sensors strategically decide when to expend energy on local model updates and when to engage in inference tasks, ultimately maximizing their long-term contribution to the network’s performance.
    
 \noindent$\bullet$ \textbf{Demonstrated Performance Gains:} Through simulations \textcolor{red}{(add simulation details later)}, we show that our integrated framework outperforms baseline approaches—such as always-on participation or simplistic energy-based selection—by achieving higher inference accuracy, lower energy consumption, and more sustainable long-term operation in EH-WSNs.


By tackling the dual challenges of sensor unreliability and energy scarcity through a rigorous game-theoretic and federated learning lens, our work addresses a critical gap in the design of sustainable, intelligent EH-WSNs. This integrated framework is theoretically grounded, yet practical for a wide range of IoT applications, from remote wildlife monitoring to large-scale industrial status tracking and precision agriculture.

The remainder of this paper is organized as follows. In Section~\ref{sec:system_model}, we present the system model, detailing the EH-WSN setup and data capture process. In Section~\ref{sec:game_theory}, we introduce the game-theoretic model of sensor participation, motivating our approach against simpler heuristics and discussing why equilibrium solutions are desirable. Section~\ref{sec:training_framework} outlines the training and fine-tuning framework that integrates the equilibrium strategies into a federated learning paradigm. Finally, Section~\ref{sec:results} presents simulation results and Section~\ref{sec:conclusion} concludes with a discussion of limitations and future work.
