\section{Appendix: Proof of Proposition 2}

In this appendix, we provide the proof of Proposition 2, establishing the existence of a pure-strategy Nash equilibrium in the inference participation game.

\subsection*{Proof of Proposition 2}

\paragraph{Setup and Definitions}

The inference participation game is defined by:

- Players: Sensors \( s_i \in \mathcal{S} \).
- Strategies: \( a_i \in \{ \text{P}, \text{NP} \} \).
- Utility functions: \( U_i(a_i, a_{-i}) \), where \( a_{-i} \) denotes the actions of other sensors.

We aim to show that the game has a pure-strategy Nash equilibrium by demonstrating that the utility functions exhibit supermodularity in \( a_i \).

\paragraph{Supermodularity and Strategic Complementarities}

A game is supermodular if the following conditions hold:

1. The strategy spaces are compact subsets of \( \mathbb{R} \) (trivially satisfied with binary strategies).
2. The utility functions \( U_i(a_i, a_{-i}) \) have increasing differences in \( (a_i, a_{-i}) \).

\emph{Increasing differences} means that for any \( a_i' > a_i \) and \( a_{-i}' > a_{-i} \),
\[
U_i(a_i', a_{-i}') - U_i(a_i, a_{-i}') \geq U_i(a_i', a_{-i}) - U_i(a_i, a_{-i}).
\]

In our context, participation (\( a_i = 1 \)) is greater than non-participation (\( a_i = 0 \)).

\paragraph{Utility Function with Increasing Differences}

The utility function for sensor \( s_i \) is
\[
U_i(a_i, a_{-i}) = \gamma \cdot a_i \cdot \Delta A_i(a_{-i}) - C_i(a_i),
\]
where \( \Delta A_i(a_{-i}) \) depends on others' actions.

Assuming \( \Delta A_i(a_{-i}) \) is increasing in \( a_{-i} \), we have that the marginal benefit of participation increases as more sensors participate. This reflects \emph{strategic complementarities}.

\paragraph{Verifying Increasing Differences}

For \( a_i' = 1 \) and \( a_i = 0 \), and \( a_{-i}' \geq a_{-i} \), we have
\begin{align*}
&U_i(1, a_{-i}') - U_i(0, a_{-i}') - [ U_i(1, a_{-i}) - U_i(0, a_{-i}) ] \\
&= \gamma [ \Delta A_i(a_{-i}') - \Delta A_i(a_{-i}) ] \geq 0,
\end{align*}
since \( \Delta A_i(a_{-i}') \geq \Delta A_i(a_{-i}) \).

Therefore, the utility functions exhibit increasing differences, satisfying the condition for supermodularity.

\paragraph{Existence of Nash Equilibrium}

By Topkis's theorem (see \cite{topkis1998supermodularity}), in supermodular games with finite strategy spaces, there exists at least one pure-strategy Nash equilibrium.

The inference participation game is supermodular, and thus, a pure-strategy Nash equilibrium exists.

\hfill \qedsymbol
