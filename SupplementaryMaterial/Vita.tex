Cyan Subhra Mishra was born in India. He received his B.Tech and M.Tech dual degree in Electronics and Communication Engineering from the National Institute of Technology, Rourkela, India, in 2016, graduating with Honors. His undergraduate research, advised by Dr.\ Sarat Kumar Patra and Tarjinder Singh (Intel), focused on hardware accelerator design.

Following his undergraduate studies, he worked as a Design Engineer at Intel in Bengaluru, India, from 2016 to 2018, where he led hardware/software co-design initiatives for machine learning accelerators on FPGA platforms. Prior to this full-time position, he completed a research internship at Intel (2015--2016) designing FPGA-based accelerators for bioinformatics algorithms.

In Fall 2018, he joined The Pennsylvania State University to pursue his Ph.D.\ in Computer Science and Engineering under the guidance of Dr.\ Mahmut Taylan Kandemir and Dr.\ Jack Sampson, with Dr.\ Chita R.\ Das and Dr.\ Vijaykrishnan Narayanan serving as co-advisors. His doctoral research focuses on enabling machine learning in power-constrained computing systems, spanning energy-harvesting wireless sensor networks, intermittent computing, and sustainable edge intelligence.

During his doctoral studies, he completed a research internship at Bell Labs, Nokia (Summer 2021), where he developed optimization strategies for autonomous ML inference serving across heterogeneous platforms. He has also served as an instructor for CMPEN 431: Introduction to Computer Architecture at Penn State, and has delivered guest lectures in graduate-level computer architecture courses.

His research has been published in premier venues including HPCA, MICRO, ISCA, ICLR, NSDI, IPDPS, PACT, ICDCS, and DATE. His work on energy-harvesting wireless sensor networks received a Best Paper Nomination at DATE 2021.

He has served as Submission Chair for IISWC 2025 and as a reviewer for IEEE Transactions on Parallel and Distributed Systems and IEEE Transactions on Computers. He is a student member of the ACM and IEEE.

\vspace{1em}
\noindent\textbf{Selected Publications:}
\begin{itemize}[leftmargin=*, itemsep=0pt]
\item C.\ S.\ Mishra, J.\ Sampson, M.\ T.\ Kandemir, V.\ Narayanan, ``Origin: Enabling On-Device Intelligence for Human Activity Recognition Using Energy Harvesting Wireless Sensor Networks,'' \textit{DATE}, 2021. \textbf{[Best Paper Nominee]}
\item C.\ S.\ Mishra, J.\ Sampson, M.\ T.\ Kandemir, V.\ Narayanan, C.\ R.\ Das, ``U\d{s}\'{a}s: A Sustainable Continuous-Learning Framework for Edge Servers,'' \textit{HPCA}, 2024.
\item C.\ S.\ Mishra, D.\ Chaudhary, J.\ Sampson, M.\ T.\ Kandemir, C.\ R.\ Das, ``Revisiting DNN Training for Intermittently-Powered Energy-Harvesting Micro-Computers,'' \textit{ICLR}, 2025.
\item C.\ S.\ Mishra, D.\ Chaudhary, M.\ T.\ Kandemir, C.\ R.\ Das, ``Salient Store: Enabling Smart Storage for Continuous Learning Edge Servers,'' \textit{PACT}, 2025.
\item J.\ R.\ Gunasekaran, C.\ S.\ Mishra, P.\ Thinakaran, B.\ Sharma, M.\ T.\ Kandemir, C.\ R.\ Das, ``Cocktail: A Multidimensional Optimization for Model Serving in Cloud,'' \textit{NSDI}, 2022.
\item K.\ Qiu, N.\ Jao, M.\ Zhao, C.\ S.\ Mishra, et al., ``ResiRCA: A Resilient Energy Harvesting ReRAM-based Accelerator for Intelligent Embedded Processors,'' \textit{HPCA}, 2020.
\end{itemize}
