The journey that culminates in this dissertation began in Fall 2018, and what followed has been six transformative years of growth, discovery, and countless moments of both struggle and triumph. As I reflect on this path, I am overwhelmed by the number of people who have shaped not just this work, but who I have become as a researcher and as a person.

I have been extraordinarily fortunate to be guided by not one or two, but four faculty mentors who each invested in me as if I were their own student: Dr.\ Mahmut Kandemir, Dr.\ Jack Sampson, Dr.\ Vijaykrishnan Narayanan, and Dr.\ Chita Das. What might appear from the outside as a challenging arrangement---navigating the expectations and directions of four busy faculty members---became instead a masterclass in constructive interference. Like the distributed systems I study in this dissertation, where redundancy ensures progress even when individual components are unavailable, my mentorship structure meant that at least one advisor was always fully engaged with my work, even when others were occupied with their many responsibilities. Different mentors took the lead during different phases of my journey, and the transitions were seamless, each handoff enriching rather than disrupting my growth.

Dr.\ Kandemir taught me how to think---not just about problems, but through them, around them, and beyond the obvious solutions that present themselves first. His patience during my moments of doubt, his unwavering belief in my potential even when I struggled to see it myself, and his ability to transform setbacks into learning opportunities have shaped me in ways that extend far beyond academia. Dr.\ Sampson brought a complementary perspective that challenged me to bridge the gap between elegant ideas and practical implementations, always pushing for rigor without losing sight of impact. Dr.\ Vijay consistently opened my eyes to new research directions, seeing connections and possibilities that I had missed entirely; his breadth of vision reminded me that the most interesting problems often lie at the intersections of fields. Dr.\ Das exemplified what it truly means to care---for research, for the department, and for students. His academic and personal wisdom, his sincere concern for my well-being, and his ever-open door made even the most challenging moments manageable. Together, these qualities fostered an environment in which intellectual curiosity thrived, ambitious research was encouraged, and I always felt supported. Through their mentorship, I learned not only how to be a dedicated researcher committed to the creation and dissemination of knowledge, but also how to be a person of strong integrity. I could not have hoped for better advisors---indeed, lifelong mentors.

I am also grateful to Dr.\ Soundar Kumara, whose guidance helped sharpen the arguments and contributions of this dissertation and whose perspective enriched my committee experience. I thank him for introducing me to the fascinating world of manufacturing and industrial engineering, and for showing me how modern industry can be sensor-driven.

No acknowledgment would be complete without mentioning Lucy Aunty, who became a mother to me in this home away from home. The wife of Dr.\ Das, she opened her heart and her home to us students in ways that words cannot adequately capture. Through countless potlucks, family gatherings, and moments of spiritual guidance, she created a sense of belonging that made State College feel less like a temporary stop and more like home. She never failed to feed us---whether it was our stomachs with her wonderful cooking or our souls with her warmth and wisdom. In a graduate journey that can often feel isolating, her presence was a constant reminder that we were part of something larger than ourselves.

To my mother, whose love knows no distance and whose strength has held our family together across continents: thank you for every sacrifice, every prayer, and every word of encouragement that carried me through the hardest moments. To my father, who passed away on December 4th, 2024---just weeks after I defended this dissertation on November 5th: I wish you could have seen this moment. You taught me to dream big, believed in my dreams before I did, and you sacrificed more than I will ever fully comprehend to give me opportunities you never had. Though the thousands of miles between us meant we could not share this journey in person, your love reached me in every late-night call, every proud word, every silent blessing. This dissertation is as much yours as it is mine. I carry you with me, always.

To my brother Sobhan, who is pursuing his own PhD in Physics at NTU Singapore: we share something that few siblings do. The peculiar loneliness of a doctoral journey---the years of dedication, the sacrifices that those outside academia cannot fully understand, the weight of pursuing knowledge at the frontier---is something we navigate together, even from opposite ends of the world. When friends who have not walked this path struggle to relate to the struggles and triumphs of graduate school, I have been blessed with a brother who understands completely. You have been my emotional anchor, my confidant, and my reminder that this journey, however isolating it sometimes feels, need not be walked alone.

I have been fortunate to collaborate with exceptional researchers who made this work richer than I could have achieved alone. Deeksha, Jashwant, and Prashanth have been incredible collaborators across multiple projects, bringing their unique perspectives and tireless dedication to our shared endeavors. Ziyu, Shulin, Haibo, and Sandeepa---with whom I have co-authored multiple papers---have been partners in the truest sense, and much of what appears in these pages bears their fingerprints. I am also grateful to Anand Sivasubramaniam, Tianyi, Wahid, Keni, Vivek, and Bikash for their contributions to various projects, and to Saleh, Abutalib, Nicholas, and Sethu for collaborations that enriched specific aspects of this research. Thank you all for giving me an opportunity to collaborate and share your wisdom.  

The HPCL, MDL and CSL labs have been my second home, and the colleagues I found there have become family. To Anup, Aditi, Ashutosh, Gulsum, Shruti, Huaipan, Sonali, Aakash, Prasanna, Iyaswarya, Siddhartha, Ashirbad, Adithya, Bala, Jihyun, Chun, Ankur, Sampad, Varun, Anusha, Woohyon, Lim, Yan, Scott, Jun, and Rishabh: thank you for the whiteboard discussions that sparked ideas, the coffee breaks that provided respite, the late nights that felt less lonely because you were there, and the camaraderie that made even the hardest days manageable. A lab is only as good as its people, and I was blessed with the best.

Beyond the lab, my dearest friends---Sanjit, Sandeep, Raman, Sourav, Anshul, Soumya, Pooja, Deeksha, Harsha, Medha, Adhi, and Ranjani---have been my support system, my escape, and my reminder that there is life beyond research. You celebrated with me, commiserated with me, and most importantly, kept me grounded. The memories we created together are among my most treasured possessions from these years.

Finally, I must acknowledge the unsung heroes who keep the department running. Erin, though now retired, deserves special thanks for years of dedicated service that made our administrative lives so much easier. The CSE IT team---Adam, Mark, Eric, and John---kept our systems running and were always ready to help, often going above and beyond their duties. I am sorry to have been a pain with so many peculiar requests. And to all the CSE staff who work behind the scenes: your contributions may not make it into publications, but they make everything else possible.

This dissertation represents not just my work, but the collective support, wisdom, and love of everyone mentioned here and many others besides. Whatever merit this work possesses, I share with all of you. Whatever errors remain are mine alone.

This research was supported in part by the National Science Foundation (NSF), the Department of Energy (DoE), the Semiconductor Research Corporation (SRC), and the Center for Brain-Inspired Computing (C-BRIC). I am grateful to these agencies for their investment in fundamental research that enables work at the intersection of machine learning, computer architecture, and sustainable computing.

In the spirit of open science and reproducibility, the source code, datasets, and artifacts associated with the research presented in this dissertation are publicly available at \url{https://github.com/cyanmishra92/}. I hope these resources prove useful to future researchers building upon this work.
