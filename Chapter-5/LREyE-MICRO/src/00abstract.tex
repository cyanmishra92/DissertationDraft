%%%%%

% Deploying deep neural networks (DNNs) in ultra-low-power remote environments
% —such as wildlife monitoring and remote sensing—
% requires hardware that is both energy-efficient and robust to environmental non-idealities. Traditional ReRAM-based processing-in-memory (PIM) architectures suffer from significant energy and latency overheads due to frequent analog-to-digital (A2D) and digital-to-analog (D2A) conversions, and are limited to computing only one neural network layer in the analog domain due to accumulated errors from analog imperfections like IR drop. In this paper, we propose and evaluate a holistic hardware-model co-design that overcomes these limitations by integrating analog-aware training techniques and analog activation functions. We model analog circuit behaviors (including IR drop, device variability, and thermal noise) into the DNN training process, enabling the network to learn compensatory mechanisms that mitigate hardware-induced errors. Furthermore, we implement activation functions using Schottky diodes, allowing computations for two consecutive neural network layers to be performed entirely in the analog domain without intermediate conversions.

% Our experimental results demonstrate that the proposed design achieves up to 68\% higher power efficiency compared to the baseline platforms and reduces energy consumption by minimizing costly A2D and D2A conversions. On benchmark datasets, our analog-aware trained models maintain high classification accuracy, achieving up to 1.63\% improvement over 4-bit quantized models. In a case study using the NTLNP wildlife image dataset, our system achieves an accuracy of 88.1\% with MicroNet, outperforming the 4-bit baseline by 4.9\% and RedEye by 6.06\%. These results highlight the effectiveness of our approach in ultra-low-power applications.
% By bridging the gap between theoretical models and practical hardware implementations, our co-design approach enables efficient and reliable neural network deployment in energy-constrained environments, paving the way for advanced edge computing applications in fields like wildlife conservation and remote sensing.




%%%%%
Deploying deep neural networks (DNNs) in ultra-low-power, intermittently powered environments---such as wildlife monitoring and remote sensing---demands hardware that is both \emph{energy-efficient} and \emph{robust} to environmental non-idealities. However, conventional ReRAM-based processing-in-memory (PIM) architectures suffer from prohibitive energy and latency overheads due to frequent analog-to-digital (A2D) and digital-to-analog (D2A) conversions, and are generally limited to computing only a single DNN layer in the analog domain. These limitations arise from accumulating errors caused by analog imperfections, including IR drops, device variability, and thermal noise. 

In this paper, we propose a holistic \emph{hardware-model co-design} that integrates analog-aware training techniques with a Schottky-diode-based analog activation function, thereby eliminating most intermediate conversions and allowing \emph{two consecutive DNN layers} to be computed entirely in the analog domain. By embedding circuit-level behavior---encompassing IR drop, device non-uniformity, and noise profiles---into the training process, our DNNs learn compensatory mechanisms to counteract hardware-induced errors. This synergy between analog hardware and model design both boosts energy efficiency and preserves high classification accuracy under stringent low-power conditions.

Experimental results show that our approach achieves up to \textbf{68\%} higher energy efficiency compared to baseline PIM systems, coupled with a \textbf{25\%} reduction in end-to-end latency. On benchmark datasets, our analog-aware trained models maintain high classification accuracy, improving up to \textbf{1.63\%} over 4-bit quantized counterparts. In a case study on the NTLNP wildlife image dataset, our system yields \textbf{88.1\%} accuracy using MicroNet, outperforming the 4-bit baseline by \textbf{4.9\%} and RedEye by \textbf{6.06\%}. Moreover, we extend these gains to modern transformer-based models, reinforcing the broad applicability of our analog activation function. By seamlessly uniting hardware constraints with DNN design, this co-design framework enables \emph{robust and efficient} DNN deployment in energy-constrained edge scenarios, paving the way for next-generation solutions in areas such as wildlife conservation and remote industrial automation.


