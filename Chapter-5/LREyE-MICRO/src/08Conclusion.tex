% % \section{Conclusion}
% % \label{sec:conclusion}

% In this paper, we have presented a holistic hardware-model co-design that significantly enhances the efficiency and robustness of ReRAM-based PIM architectures for DNN inference in ultra-low-power, energy-harvesting environments. By integrating analog-aware training techniques and implementing activation functions directly in the analog domain using Schottky diodes, we addressed the limitations of traditional architectures that suffer from energy and latency overheads due to frequent A2D and D2A conversions and are restricted to computing only one layer in the analog domain.

% Our analog-aware training methodology models analog circuit behaviors—including IR drop, device variability, and thermal noise—into the DNN training process. This enables the neural network to learn compensatory mechanisms that mitigate hardware-induced errors, maintaining high classification accuracy despite analog non-idealities. The implementation of diode-based activation functions allows computations for two consecutive neural network layers to be performed entirely in the analog domain without intermediate conversions, thus reducing energy consumption and latency.

% The experimental results demonstrate that our proposed design achieves up to 68\% higher power efficiency compared to the baseline platforms. On benchmark datasets, our analog-aware trained models maintained high classification accuracy, achieving up to 1.63\% improvement over 4-bit quantized models. In the case study using the NTLNP~\cite{ntlnpdataset, ntlnprepo} wildlife image dataset, our system achieved an accuracy of 88.1\% with MicroNet~\cite{li2021micronet}, outperforming the 4-bit baseline by 4.9\% and RedEye~\cite{RedEye2020} by 6.06\%. These results validate the effectiveness of our approach in real-world energy-constrained applications.

% By bridging the gap between theoretical models and hardware implementations, our work enables efficient and reliable DNN deployment in environments with stringent energy constraints, paving the way for advanced edge computing applications like wildlife conservation, remote sensing, and industrial automation.
% %Reducing the dependency on A2D and D2A conversions and enhancing robustness to analog non-idealities paves the way for advanced edge computing applications in fields like wildlife conservation, remote sensing, and industrial automation.

\section{Conclusions}
\label{sec:conclusions}
In this paper, we have introduced a hardware-software co-design that tackles the inherent challenges of ReRAM-based PIM for ultra-low-power, energy-harvesting edge deployments. By integrating a diode-based activation circuit with IR-drop and noise-aware training, our approach enables two consecutive DNN layers to be executed entirely in the analog domain, thus reducing repeated ADC/DAC overhead and enhancing power efficiency. The proposed methodology incorporates realistic device and circuit variations, ensuring that the network learns to compensate for IR drops, diode forward voltage, and thermal or flicker noise.

Experimental results across a range of popular DNNs, including MobileNetV3-Small, EfficientNet-Lite0, and MCUNet, show that our design achieves significant improvements in power efficiency—often 20–50\% higher (or more) than state-of-the-art analog accelerators like ISAAC~\cite{shafiee2016isaac}, ResiRCA~\cite{qiu2020resirca}, and Nebula~\cite{singh2020nebula}—while retaining competitive accuracy under 4-bit quantization. In a comprehensive wildlife monitoring case study on the NTLNP dataset~\cite{ntlnpdataset, ntlnprepo}, our system reaches 95.8\% accuracy with MobileNetV3-Small and 88.1\% with EfficientNet-Lite0 under variable solar power conditions, outperforming baseline 8-bit and 4-bit platforms. These gains reflect how hardware-aware training, grounded in circuit-level modeling of diode activations and crossbar non-idealities, enables robust, high-efficiency inference in intermittent power scenarios.

Overall, this work demonstrates that bridging analog circuit physics and deep learning models can unlock notable energy and latency savings without compromising performance. By addressing ADC/DAC overheads, device variability, and noise through a unified training framework, we pave the way for resource-efficient DNN deployment in remote or battery-operated environments such as wildlife monitoring, environmental sensing, and industrial IoT.
