%%
%% This is file `sample-sigconf.tex',
%% generated with the docstrip utility.
%%
%% The original source files were:
%%
%% samples.dtx  (with options: `all,proceedings,bibtex,sigconf')
%% 
%% IMPORTANT NOTICE:
%% 
%% For the copyright see the source file.
%% 
%% Any modified versions of this file must be renamed
%% with new filenames distinct from sample-sigconf.tex.
%% 
%% For distribution of the original source see the terms
%% for copying and modification in the file samples.dtx.
%% 
%% This generated file may be distributed as long as the
%% original source files, as listed above, are part of the
%% same distribution. (The sources need not necessarily be
%% in the same archive or directory.)
%%
%%
%% Commands for TeXCount
%TC:macro \cite [option:text,text]
%TC:macro \citep [option:text,text]
%TC:macro \citet [option:text,text]
%TC:envir table 0 1
%TC:envir table* 0 1
%TC:envir tabular [ignore] word
%TC:envir displaymath 0 word
%TC:envir math 0 word
%TC:envir comment 0 0
%%
%%
%% The first command in your LaTeX source must be the \documentclass
%% command.
%%
%% For submission and review of your manuscript please change the
%% command to \documentclass[manuscript, screen, review]{acmart}.
%%
%% When submitting camera ready or to TAPS, please change the command
%% to \documentclass[sigconf]{acmart} or whichever template is required
%% for your publication.
%%
%%
\documentclass[sigconf, screen, review]{acmart}
%\usepackage{etex}
%%%%% Packages and Commands by Cyan %%%%%

\usepackage[utf8]{inputenc} % allow utf-8 input
\usepackage[T1]{fontenc}    % use 8-bit T1 fonts
\usepackage{hyperref}       % hyperlinks
\usepackage{url}            % simple URL typesetting
\usepackage{booktabs}       % professional-quality tables
\usepackage{amsfonts}       % blackboard math symbols
\usepackage{nicefrac}       % compact symbols for 1/2, etc.
\usepackage{microtype}      % microtypography
\usepackage{xcolor}         % colors


\usepackage{multirow}
\usepackage{wrapfig}

\usepackage{graphics}
\usepackage{graphicx}
\usepackage{amsmath}
\usepackage{wrapfig}
\usepackage[subfigure]{tocloft}
\usepackage{subcaption}
\usepackage{amsfonts,bm}
\usepackage{natbib} 

\usepackage{macros}
% \usepackage{algorithm}
% \usepackage{algorithmic}
% \usepackage{float}  % For math environments and symbols Ensure the 'H' position specifier is available
\usepackage{algorithm}

%\usepackage{algorithmic}

\usepackage{algpseudocode}
\usepackage{amsmath}
\usepackage{float}
\usepackage{multicol}
\usepackage{setspace}

\usepackage{tikz}
\usetikzlibrary{shapes.geometric, arrows}
\usepackage{circuitikz}
 \usepackage{tabularray}
 \usepackage{multirow}

% \usepackage{etex}

\usepackage[inline]{enumitem}

%%
%% \BibTeX command to typeset BibTeX logo in the docs
\AtBeginDocument{%
  \providecommand\BibTeX{{%
    Bib\TeX}}}

%% Rights management information.  This information is sent to you
%% when you complete the rights form.  These commands have SAMPLE
%% values in them; it is your responsibility as an author to replace
%% the commands and values with those provided to you when you
%% complete the rights form.
\setcopyright{none} 
%\setcopyright{acmlicensed}
\copyrightyear{2025}
\acmYear{2025}
\acmDOI{XXXXXXX.XXXXXXX}

%% These commands are for a PROCEEDINGS abstract or paper.
\acmConference[MICRO 2025]{The 58th IEEE/ACM International Symposium on Microarchitecture}{October 18--22, 2025}{Seoul, Korea}
%%
%%  Uncomment \acmBooktitle if the title of the proceedings is different
%%  from ``Proceedings of ...''!
%%
%%\acmBooktitle{Woodstock '18: ACM Symposium on Neural Gaze Detection,
%%  June 03--05, 2018, Woodstock, NY}
\acmISBN{978-X-XXXX-XXXX-X/XX/XX}


%%
%% Submission ID.
%% Use this when submitting an article to a sponsored event. You'll
%% receive a unique submission ID from the organizers
%% of the event, and this ID should be used as the parameter to this command.
%%\acmSubmissionID{123-A56-BU3}

%%
%% For managing citations, it is recommended to use bibliography
%% files in BibTeX format.
%%
%% You can then either use BibTeX with the ACM-Reference-Format style,
%% or BibLaTeX with the acmnumeric or acmauthoryear sytles, that include
%% support for advanced citation of software artefact from the
%% biblatex-software package, also separately available on CTAN.
%%
%% Look at the sample-*-biblatex.tex files for templates showcasing
%% the biblatex styles.
%%

%%
%% The majority of ACM publications use numbered citations and
%% references.  The command \citestyle{authoryear} switches to the
%% "author year" style.
%%
%% If you are preparing content for an event
%% sponsored by ACM SIGGRAPH, you must use the "author year" style of
%% citations and references.
%% Uncommenting
%% the next command will enable that style.
%%\citestyle{acmauthoryear}
\settopmatter{printfolios=true}
\settopmatter{printacmref=false}
%\pagestyle{plain}

%%
%% end of the preamble, start of the body of the document source.
\begin{document}

%%%%%%%%%%%%%%%%%%%%%%%%%%%%%%
%--------------------------------------------------------------------------
% 1. COVER PAGE (same format as the ACM two-column style)
%--------------------------------------------------------------------------
% \twocolumn[
%         \centering
%         {\LARGE \bf Revision Summary: Hardware-Aware Neural Network Co-Design with Analog Activation for Energy-Efficient ReRAM Crossbars}\\
%         \large ISCA 2025 Submission
%     \textbf{\#737}
        
        
% ]
% %%%%% %%%%% %%%%%
Thank you for the constructive and insightful feedback on our manuscript. We have revised the paper to address the reviewers’ concerns regarding Section~\ref{sec:eval}’s organization, the evaluation of modern DNN models, and clarity on circuit-level details. We summarize the main revisions and respond to the core concerns.
%, and clarify both what we have done and why our approach makes sense.

\section*{Revision Summary}
We restructured Section~\ref{sec:eval} to provide a clearer flow of the experimental setup, figures, and baseline comparisons. We now consistently label subplots, unify color schemes, and explicitly reference each bar or marker in the text to eliminate any ambiguity regarding the improvements presented. In addition, we evaluated our design on transformer models offering a more contemporary evaluation beyond LeNet and MicroNet. We also elaborate on how we achieve the claimed 25\% latency saving by skipping an ADC/DAC cycle after every second layer, leading to a substantial reduction in end-to-end inference time for multilayer networks. We added explanations on trainable parameters, quantization and driver circuit details to further clarify the concerns. 

\section*{General Concerns}
Some reviewers noted that our dual emphasis on analog circuit design and model training could appear more machine learning centric than architectural. To address this, we clarified in Sections~\ref{sec:Activation}--\ref{sec:SystemIntegration} how tile sizing, quantization, domain conversion, and I/O overheads are central to an ISCA-level architectural discussion. We refined the circuit descriptions to explicitly indicate when the ReRAM columns output current, how transimpedance amplifiers convert this current to a voltage domain for Schottky diode thresholding, and how multi-bit cell variability, environmental impact, and other noises are managed by modeling device mismatches as random variables during training rather than by physically modifying hardware. We believe these revisions not only improve clarity, but also reinforce the architectural rigor and motivation behind our design choices.

\section*{Specific Questions}
\noindent
\textbf{Reviewer A.}  
We have majorly restructured Section~\ref{sec:eval} to improve readability, unify the legends of the figures, and ensure that every sub-figure, bar and marker is clearly referenced in the text. Figures that previously lacked color consistency are now reorganized with textures for better visualizations and text-to-figure corelation.

\noindent
\textbf{Reviewer B.}  
We clarify our use of a 4-bit quantization baseline, arguing that many low-power ReRAM PIM systems adopt a similar small bit-width quantization to minimize overhead and reduce data movement costs. We also explicitly break down the 25\% latency (Page-10, Figure~\ref{fig:pipelineLatency}) reduction—attributable to the removal of one ADC/DAC cycle every two layers—and discuss how our approach can be extended to more advanced transformer based models (Page--13, Section~\ref{sec:transformerEval}) through additional analog fine-tuning/retraining.
%, which we leave for future work. 

\noindent
\textbf{Reviewer C.}  
We cite and contrast LeCA (ISCA 2023), emphasizing that although the in-sensor compressive acquisition differs from our crossbar-based diode activation, both approaches leverage hardware-aware training to improve robustness. We have added a discussion to explain how sampling device parameters for temperature, power fluctuation, and IR drop during training enables a single retrained model to operate reliably under diverse conditions (Pages--6, 7, and Algorithm~\ref{alg:unified_training}).

\noindent
\textbf{Reviewer D.}  
We include evaluations on modern CNNs, such as MobileNetV2 and SqueezeNet, and clarify the comparisons between baseline FP32 and 4-bit performance~\ref{fig:accuracy_comparison}. We note that only ReLU is natively supported in the analog domain, while other activation functions (e.g., Softmax, Sigmoid) are performed digitally. The circuit design could be augmented (e.g. with clipper and clamper circuits) to represent leaky-Relu, and Relu6 etc. We justify limiting analog processing to two layers by highlighting the effects of IR drop accumulation, acknowledging that additional analog amplification might allow more layers at increased cost (Page--7). %For the case study in Section~6.4, we provide further details on variable solar power conditions and associated energy consumption. 
                              
\noindent
\textbf{Reviewer E.}  
We have updated the circuit explanation in Section~\ref{sec:Activation} and Figure~\ref{fig:DiodeDriver}, clarifying that the crossbar output is a current, which is then converted to a voltage domain by a transimpedance amplifier. We show that parameters such as $V_{\text{th}}$, $\alpha$, and the IR drop factor $\gamma$ are sampled during training to model device variability, while remaining fixed in hardware (Page-7, and new Algorithm~\ref{alg:unified_training}). 

Reprogramming ReRAM cells for large models is primarily a technology-dependent process, 
involving voltage pulses and program--verify loops to set each cell's resistance levels. Recent works~\cite{IMBNature} shows that the reprogramming cost/tile could be $\le 110pJ$. Our tile-based design supports layer-by-layer loading when the entire model (e.g., MobileNetV2) cannot fit into on-chip ReRAM simultaneously.

%We also address multi-bit ReRAM noise modeling, define SNR clearly, and include the overhead of weight reloading for larger networks like SqueezeNet and MobileNetV2. The revised Section~6 now contains improved figures with consistent legends to eliminate any confusion about which data correspond to which approach.

\section*{Modern Networks and Hardware}
Our work majorly focuses on performing popular ReLU activation entirely on analog domain, and incorporating hardware aware training for ReRAM xBars. Therefore, we believe, any hardware optimizations proposed in newer works~\cite{wu2024autohet, ma2025sipt} could be integrated into our design. To showcase generalization of the proposed activation function (Equation~\ref{eqn:DiodeActivation}), we evaluated the software only deployment of modern transformer based models~\cite{jiao2019tinybert, setyawan2025microvit} with the proposed activation function at different quantization levels (Page--13, Section~\ref{sec:transformerEval}). The results (Table~\ref{tab:transformerAcc1}) shows neglegible accuracy loss compared to the baseline.

\section*{Relevance for ISCA}
Our work addresses a critical architectural challenge: integrating analog processing-in-memory with hardware-aware training to overcome non-idealities such as IR drop, diode variability, and supply fluctuations. By combining a techniques like tile dropout with innovative analog circuit design (Schottky diode activation and reduced ADC/DAC conversions), we demonstrate up to 66\% higher power efficiency and a 25\% reduction in latency compared to existing ReRAM PIM baselines. The amalgamation of hardware-software-model co-design approach align directly with ISCA’s focus on specialized accelerators, energy-efficient designs, and emerging memory technologies. We believe that the revised version of our paper
%—featuring enhanced clarity and evaluations on modern netowrks—
provides a compelling and well-motivated case for advancing the state of ultra-low-power ReRAM-based systems.


%%%%% %%%%% %%%%%

%%%%% %%%%% %%%%%
\clearpage
% \textcolor{blue}{
% THE TEXT BELOW ARE SOME ADDITIONAL CLARIFICATIONS AND CAN BE INTEGRATED ABOVE FOR ADDITIONAL CLARIFICATIONS.
% }

% \begin{verbatim}
% %%%% %%%%% %%%%% %%%% %%%%% %%%%% %%%% %%%%% %%%%%
% \end{verbatim}

% \section*{Additional Clarifications on Our Design Choices}
% \noindent
% \textbf{Focus on 4-bit Quantization.}  
% We select a 4-bit baseline because it strikes an optimal balance between achieving acceptable model accuracy and meeting the extremely tight energy and memory constraints typical in remote deployments. Although higher precision may yield better accuracy, it also increases the overhead of DAC/ADC conversions, undermining our goal of reducing energy consumption through minimized domain transfers.


% \noindent
% \textbf{Two Consecutive Layers in Analog.}  
% Due to cumulative effects such as IR drop, diode voltage drop, and analog noise, chaining more than two layers in the analog domain can lead to unacceptable signal degradation. Our empirical evaluations indicate that processing two consecutive layers in analog delivers significant ADC/DAC savings—reducing overall latency by approximately 25\%—without incurring a prohibitive accuracy loss. Future work may explore analog amplification techniques to support additional layers, but our current design presents a robust and practical compromise.


% \noindent
% \textbf{Diode-Based ReLU Implementation.}  
% A Schottky diode’s low forward voltage and rapid switching make it an ideal candidate for an analog approximation of the ReLU function. This implementation obviates the need for off-chip activation computations, thus reducing both power consumption and latency. We incorporate diode parameters (e.g., $V_\text{th}$, $\alpha$) as random variables during training to capture device variability, ensuring that the model remains robust even under non-ideal real-world conditions.


% \noindent
% \textbf{Model-Hardware Co-Design.}  
% Our approach integrates analog non-idealities—including IR drop, additive noise, and multi-bit device variability—directly into the training pipeline. This allows the neural network to learn compensatory mechanisms, achieving higher energy efficiency and robustness compared to conventional digital designs that require frequent transfers back to the MCU domain.


% \noindent
% \textbf{Tile Architecture and Scheduling.}  
% We employ a $32 \times 32$ crossbar tile architecture that balances manageable IR drops with high computational parallelism. Our scheduling strategy processes two consecutive layers within the same tile, thereby minimizing data transfers and maximizing the benefits of in-situ analog operations. This design is scalable, as additional tiles can be added or weight reloading optimized to accommodate larger networks or deeper models.


% \noindent
% \textbf{Why Our Approach Makes Sense.}  
% By reducing the number of analog--digital conversions and incorporating hardware imperfections into the training process, our design simultaneously enhances power efficiency and maintains high accuracy across varying operational environments. The diode-based activation seamlessly integrates with the analog multiply--accumulate operations of ReRAM crossbars, enabling us to skip at least one conversion cycle per two layers. In ultra-low-power deployments—such as wildlife or industrial monitoring—this leads to significant improvements (up to 66\% higher power efficiency and 25\% latency savings) over traditional digital or single-layer analog designs.
% %%%%% %%%%% %%%%%

%%%%% %%%%% %%%%%

% %%%%% %%%%% %%%%%
% Thank you for the constructive and insightful feedback on our manuscript. We have significantly revised the paper to address the reviewers’ concerns about Section~6’s organization, the evaluation of modern DNN models, clarity on circuit-level details, and deeper comparisons to state-of-the-art ReRAM-based platforms. Below, we summarize the main revisions, respond to the core issues raised, and clarify why this work is best suited for ISCA.

% \section*{Revision Summary}
% We restructured Section~6 to provide a clearer flow of the experimental setup, figures, and baseline comparisons. We now consistently label subplots, unify color schemes, and explicitly reference each bar or marker in the text to eliminate confusion regarding which improvements are shown. In addition, we expanded our benchmarks to include SqueezeNet and MobileNetV2, offering a more contemporary evaluation beyond LeNet and MicroNet. We also elaborate on how we achieve the claimed 25\% latency saving, explaining that skipping an ADC/DAC cycle after every second layer leads to an end-to-end reduction in inference time for multi-layer networks. Finally, we incorporated references to works like LeCA (ISCA 2023) and recent ReRAM accelerators to position our work relative to in-sensor methods and advanced crossbar designs.

% \section*{General Concerns}
% Some reviewers noted the work’s dual emphasis on analog circuit design and model training could appear more machine-learning-centric than architectural. To address this, we clarified in Sections~3–5 how tile sizing, scheduling logic, partial sum reloading, and domain conversion overheads are central to an ISCA-level architecture discussion. We likewise refined the circuit descriptions, ensuring that it is explicit when the ReRAM columns output current, how it converts to a voltage domain for the Schottky diode threshold, and how we handle multi-bit cell variability in training. Furthermore, we detailed that we only “train” these parameters in software as random variables reflecting device mismatch, not physically change them on chip.

% \section*{Specific Questions}
% \noindent
% \textbf{Reviewer A.} We have thoroughly rewritten Section~6 to improve readability, unify figure legends, and ensure each sub-figure, bar, and marker is clearly referenced in the text. Figures that previously lacked color consistency or partial legends are now reorganized so that the paragraph explanations match precisely with what is shown.

% \noindent
% \textbf{Reviewer B.} In clarifying the 4-bit baseline, we argue that many low-power ReRAM PIM systems adopt 4-bit or similar small bit-width quantization to minimize overhead and data movement costs. We also explicitly break down our 25\% latency reduction, attributing it to removing one ADC/DAC step after each second layer. We discuss how our approach could extend to larger or more advanced models (e.g., transformers), although these remain future work.

% \noindent
% \textbf{Reviewer C.} We cite and contrast LeCA (ISCA 2023), emphasizing that while in-sensor compressive acquisition differs from our crossbar-based diode activation, both use hardware-awareness in training. We expand the “extreme environments” discussion, explaining how sampling device parameters for temperature, power fluctuation, and IR drop during training allows one retrained model to handle wide operating conditions in practice.

% \noindent
% \textbf{Reviewer D.} We now evaluate modern CNNs, such as MobileNetV2 and SqueezeNet, clarifying how we measure baseline FP32 vs. 4-bit performance. We also note that only ReLU is natively supported in analog, but final layers (Softmax, Sigmoid) can be done digitally. We justify limiting analog processing to two layers by referencing IR drop accumulation, though additional amplification could enable more layers at added cost. For the case study (Section~6.4), we add more detail on variable power conditions and the associated energy consumption.

% \noindent
% \textbf{Reviewer E.} We provide an updated circuit explanation in Section~3, clarifying that the crossbar output is current, which is then converted to a voltage domain for diode thresholding. We show how parameters like $V_{\text{th}}$, $\alpha$, and the IR-drop factor $\gamma$ are sampled during training but not physically altered in hardware. We also address multi-bit ReRAM noise modeling, SNR definitions, and the overhead of weight reloading for larger networks such as SqueezeNet and MobileNetV2. The final revised Section~6 includes improved figures with consistent legends to address confusion about which data correspond to which approach.

% \section*{Relevance for ISCA}
% Our work tackles a prominent architecture challenge: how to integrate analog processing-in-memory with domain-specific training to overcome non-idealities such as IR drop, diode variability, and partial supply fluctuations. By bridging a tile-based scheduling methodology (tile dropout, partial weight reloading) and analog circuit design (Schottky diode activation, skipping frequent ADC/DAC conversions), we achieve up to 66\% higher power efficiency and 25\% latency savings over ReRAM PIM baselines. These findings resonate with ISCA’s focus on specialized accelerators, energy-efficient designs, and emerging memory technologies. We believe the revised version of this paper demonstrates both the architectural rigor and the real-world practicality needed to advance the state of ultra-low-power ReRAM-based systems.
% %%%%% %%%%% %%%%%  % Your custom cover text (formatted as normal LaTeX inside a two-column layout)

% \clearpage  % Ensure the next page starts fresh
%%%%%%%%%%%%%%%%%%%%%%%%%%%%%%

%%
%% The "title" command has an optional parameter,
%% allowing the author to define a "short title" to be used in page headers.
\title{Hardware-Aware Neural Network Co-Design with Analog Activation for Energy-Efficient ReRAM Crossbars}
\subtitle{\normalsize{MICRO 2025 Submission
    \textbf{\#1200} -- Confidential Draft -- Do NOT Distribute!!}}
%%
%% The "author" command and its associated commands are used to define
%% the authors and their affiliations.
%% Of note is the shared affiliation of the first two authors, and the
%% "authornote" and "authornotemark" commands
%% used to denote shared contribution to the research.
%\author{\normalsize{ISCA 2025 Submission
 %   \textbf{\#NaN} -- Confidential Draft -- Do NOT Distribute!!}}

%%
%% By default, the full list of authors will be used in the page
%% headers. Often, this list is too long, and will overlap
%% other information printed in the page headers. This command allows
%% the author to define a more concise list
%% of authors' names for this purpose.

%%
%% The abstract is a short summary of the work to be presented in the
%% article.
\begin{abstract}
There is an increasing demand for intelligent processing on ultra-low-power internet of things (IoT) device. Recent works have shown substantial efficiency boosts by executing inferences directly on the IoT device (node) rather than transmitting data. However, the computation and power demands of Deep Neural Network (DNN)-based inference pose significant challenges in an energy-harvesting wireless sensor network (EH-WSN). Moreover, these tasks often require responses from multiple physically distributed EH sensor nodes, which impose crucial system optimization challenges in addition to per-node constraints. To address these challenges, we propose \emph{Seeker}, a hardware-software co-design approach for increasing on-sensor computation, reducing communication volume, and maximizing inference completion, without violating the quality of service, in EH-WSNs coordinated by a mobile device. \emph{Seeker} uses a \emph{store-and-execute} approach to complete a subset of inferences on the EH sensor node, reducing communication with the mobile host. Further, for those inferences unfinished because of the  harvested energy constraints, it leverages task-aware coreset construction to efficiently communicate compact features to the host device. We evaluate  \emph{Seeker} for human activity recognition, as well as predictive maintenance and show $\approx 8.9\times$ reduction in communication data volume with $86.8\%$ accuracy, surpassing the $81.2\%$ accuracy of the state-of-the-art.

%%%%

%{\bf KEEP ALL DECIMAL POINTS TO ONE DIGIT}
%\textcolor{red}{(one line for result)}
\end{abstract}

%%
%% The code below is generated by the tool at http://dl.acm.org/ccs.cfm.
%% Please copy and paste the code instead of the example below.
%%
%\begin{CCSXML}
%<ccs2012>
% <concept>
%  <concept_id>00000000.0000000.0000000</concept_id>
%  <concept_desc>Do Not Use This Code, Generate the Correct Terms for Your Paper</concept_desc>
%  <concept_significance>500</concept_significance>
% </concept>
% <concept>
%  %<concept_id>00000000.00000000.00000000</concept_id>
%  <concept_desc>Do Not Use This Code, Generate the Correct Terms for Your Paper</concept_desc>
%  <concept_significance>300</concept_significance>
% </concept>
% <concept>
%  %<concept_id>00000000.00000000.00000000</concept_id>
%  <concept_desc>Do Not Use This Code, Generate the Correct Terms for Your Paper</concept_desc>
%  <concept_significance>100</concept_significance>
% </concept>
% <concept>
 % <concept_id>00000000.00000000.00000000</concept_id>
%  <concept_desc>Do Not Use This Code, Generate the Correct Terms for Your Paper</concept_desc>
%  <concept_significance>100</concept_significance>
% </concept>
%</ccs2012>
%\end{CCSXML}

%\ccsdesc[500]{Do Not Use This Code~Generate the Correct Terms for Your Paper}
%\ccsdesc[300]{Do Not Use This Code~Generate the Correct Terms for Your Paper}
%\ccsdesc{Do Not Use This Code~Generate the Correct Terms for Your Paper}
%\ccsdesc[100]{Do Not Use This Code~Generate the Correct Terms for Your Paper}

%%
%% Keywords. The author(s) should pick words that accurately describe
%% the work being presented. Separate the keywords with commas.
% \keywords{Do, Not, Us, This, Code, Put, the, Correct, Terms, for,
%   Your, Paper}
%% A "teaser" image appears between the author and affiliation
%% information and the body of the document, and typically spans the
%% page.
% \begin{teaserfigure}
%   \includegraphics[width=\textwidth]{sampleteaser}
%   \caption{Seattle Mariners at Spring Training, 2010.}
%   \Description{Enjoying the baseball game from the third-base
%   seats. Ichiro Suzuki preparing to bat.}
%   \label{fig:teaser}
% \end{teaserfigure}

%\received{20 February 2007}
%\received[revised]{12 March 2009}
%\received[accepted]{5 June 2009}

%%
%% This command processes the author and affiliation and title
%% information and builds the first part of the formatted document.
\maketitle


% \section{Introduction}
% \label{sec:introduction}
% %%%%% %%%%% %%%%%
% The proliferation of Internet of Things (IoT) devices in diverse and challenging environments—such as wildlife monitoring~\cite{tuia2022perspectives,gobieski2019intelligence}, deep mine surveillance~\cite{bai2020real, ni2024detection}, and remote industrial automation~\cite{visconti2024machine, khalil2021deep}—demands efficient and reliable image classification~\cite{RedEye2020, qiu2020resirca}. These applications often operate under stringent energy constraints, requiring ultra-low power consumption, long device lifetimes, and minimal maintenance~\cite{qiu2020resirca, gobieski2019intelligence}. For example, wildlife monitoring sensors must often run for years without battery replacements, and deep-mine surveillance systems face hostile environments where frequent maintenance is infeasible. Energy-harvesting technologies are increasingly being leveraged to meet these requirements, enabling devices to operate indefinitely, but intermittently~\cite{Lv2022, qiu2020resirca, ma2017incidental}.  In such contexts, traditional image classification solutions fall short due to their significant energy overhead~\cite{RedEye2020, IMBNature}, necessitating novel paradigms for energy-efficient computing.

% Amongst these approaches, Processing-in-Memory (PIM) architectures have emerged as a transformative solution, 
% %for these constraints
% with Resistive RAM (ReRAM)--based crossbars (xBars) demonstrating particular promise, especially for DNN-based inference serving~\cite{shafiee2016isaac}. xBars excel in their ability to perform matrix-vector multiplications (MVMs) directly in the analog domain, drastically reducing the energy overhead of data movement and digital processing~\cite{reramNature, chi2016prime, shafiee2016isaac, singh2020nebula}. A typical image inference pipeline using ReRAM crossbars is depicted in Figure~\ref{fig:IntroDiagram}. Data from image sensors are stored in the memory (often an eDRAM)~\cite{singh2020nebula, shafiee2016isaac, chi2016prime} and part of the multiply-accumulate functionality of the DNN execution is performed in the crossbar. 
% Finally, the activation functions and data-flow orchestration are typically managed by an microcontroller unit (MCU) or a dedicated control block.

% Furthermore, the non-volatility and endurance of ReRAM make it particularly suitable for low-power scenarios, where reliability over extended lifetimes is paramount~\cite{chen2020reram, rizk2019demystifying}. 
% In addition, compatibility with existing CMOS fabrication processes further enhances the commercial viability of ReRAM-based systems~\cite{IMBNature}.

% \begin{figure}[t]
%     \centering
%     \includegraphics[width=\linewidth]{figs/RichIntroDiagram.pdf}
%     \vspace{-22pt}\caption {An example wildlife classification deployment - Analog image sensor data is stored in digital memory, then  processed in crossbar arrays in the mixed domain. The MCU, in the digital domain, performs activation functions. Every DNN layer undergoes expensive domain transfers.}
%     \label{fig:IntroDiagram}
% \end{figure}


% Despite these advantages, significant challenges remain in deploying ReRAM-based systems for ultra low-power and energy-harvesting environments. A critical limitation is the energy and latency overheads associated with analog-to-digital (A2D) and digital-to-analog (D2A) conversions~\cite{shafiee2016isaac, singh2020nebula}. As depicted in Figures~\ref{fig:IntroDiagram} and~\ref{fig:xbarFlow}, in a typical image inference flow, the matrix-vector multiplication (MVM) computations happen in the crossbar arrays operating in the analog domain, while the activation functions, control management, and memory management are performed in the digital domain~\cite{shafiee2016isaac, singh2020nebula, chi2016prime, qiu2020resirca}. After performing MVM, each layer in a DNN typically needs to execute a nonlinear activation function, particularly the rectified linear unit (ReLU)~\cite{lecun1998gradient, hahnloser2000digital, hahnloser2000correction}, which is indispensable for modern neural networks~\cite{howard2017mobilenets}. Although activation computations are generally inexpensive, they are typically done off-crossbar and require A2D conversion~\cite{shafiee2016isaac}, leading to frequent A2D and D2A conversions. These conversions can consume up to $82\%$ of the power and $31\%$ of the area~\cite{shafiee2016isaac, chi2016prime, IMBNature}, severely undermining the efficiency gains of the PIM architecture. Traditional designs often offload these activation computations to external digital processors, such as microcontroller units (MCUs)~\cite{singh2020nebula, qiu2020resirca} or specialized digital circuits~\cite{IMBNature}, which not only incurs additional energy and latency costs but also undermines the core advantage of PIM architectures by requiring frequent data movement between analog crossbars and digital processing units. For instance, each off-crossbar activation processing step can add a latency overhead of $15\%$ to $18\%$ and contribute approximately $5\%$ to the total energy budget in ultra-low-power systems~\cite{shafiee2016isaac, chi2016prime, IMBNature}.

% Another significant issue is the degradation of computational accuracy due to IR drops and other analog noise/variability in large-scale crossbar arrays. Since the crossbar computation is entirely based on Ohmic properties, the voltage drop across stages causes an uneven distribution of voltage within the crossbar. This leads to reduced fidelity in analog computations, with accuracy drops of up to $25\%$ reported in certain configurations~\cite{IRDropDAC, IRDropTCAD, lee2024mitigating}. 
% Mitigation strategies, such as current balancing and the inclusion of intermediate storage registers, exacerbate energy consumption, often increasing the overall power budget~\cite{crafton2022characterization, IRDropICCD}. Furthermore, traditional neural network models are often designed without consideration of the non-idealities inherent in analog hardware. Hardware-induced noise, variability in device characteristics, and diverse operating conditions can lead to significant performance degradation, particularly in scenarios with tight energy budgets and low data fidelity~\cite{lee2024mitigating, IRtrainingBL}. For example, analog noise can introduce errors that degrade classification accuracy by $15\%$ if not properly accounted for in the network design~\cite{singh2020nebula}.

% To address these challenges, we advocate for the co-optimization of model architectures and hardware configurations, a strategy that remains underexplored. Current designs often fail to consider how tile sizes, data representations, and hardware resources interact to impact overall energy efficiency and performance in low-power xBars. Although recent work has explored adaptive tiling strategies~\cite{qiu2020resirca}, they lack joint optimization of neural networks and PIM architectures. Such co-design strategies are critical for unlocking the full potential of ReRAM-based accelerators in ultra-low-power and energy-harvesting systems.
% In this paper, we address these challenges by proposing a novel framework that integrates \emph{hardware-aware neural network design} and \emph{energy-efficient activation strategies}. Our approach bridges the gap between hardware limitations and neural network requirements, delivering practical solutions for deploying ReRAM-based systems in energy-constrained environments. Our experimental results demonstrate up to \textbf{68\%} improvement in energy efficiency and \textbf{25\%} reduction in latency, allowing robust image classification for applications such as wildlife monitoring and industrial automation under strict energy constraints. The \textbf{key contributions} of this paper are as follows:

% \noindent$\bullet$ \textbf{Analog Activation with Schottky Diodes:} We present a novel analog activation function implemented using Schottky diodes, which offer low voltage drop, rapid switching and minimal leakage. This eliminates the need for off-chip or off-crossbar digital activation processing, reducing energy consumption and latency while ensuring seamless compatibility with analog crossbar operations.

% \noindent$\bullet$ \textbf{Hardware-Aware Training:} Our training pipeline incorporates real-world hardware non-idealities to enhance robustness and efficiency. We introduce an activation function that models Schottky diode behavior in the analog domain, enabling the neural network to adapt to hardware constraints during training. We also design a loss function that accounts for IR drop and hardware noise profiles, resulting in models resilient to voltage fluctuations and analog noise—critical factors in ultra low-power applications.

% \noindent$\bullet$ \textbf{Model-Hardware Co-Design:} We adopt a co-design strategy that simultaneously optimizes the DNN and hardware architecture for superior energy efficiency and performance. Key innovations include a tile-based architecture that allows parallel analog computations within crossbar arrays and the ability to compute two consecutive neural network layers entirely in the analog domain. Additionally, during training, techniques such as tile dropout are employed to boost model robustness against real-world challenges like incomplete or noisy data.

% \noindent$\bullet$ \textbf{Comprehensive Evaluation:} We conduct extensive evaluations to validate the proposed system. Our results demonstrate significant improvements over state-of-the-art methods, achieving up to \textbf{68\%} energy savings and \textbf{25\%} latency reductions while maintaining high classification accuracy. Specifically, on benchmark datasets, our analog-aware trained models maintain high classification accuracy, achieving up to \textbf{1.63\%} improvement over 4-bit quantized models. In a case study using the NTLNP wildlife image dataset~\cite{ntlnpdataset, tan2022animal, ntlnprepo}, our system achieves an accuracy of \textbf{88.1\%} with MicroNet, outperforming the 4-bit baseline by \textbf{4.9\%} and RedEye by \textbf{6.06\%}, highlighting the versatility and practicality of our solution. \textcolor{blue}{We further expand our evaluations on modern transformer based architectures to show generalization of the new activation function.}

% \textcolor{blue}{To the best of our knowledge, this is the first work to integrate hardware-aware training, 
% an analog activation circuit based on Schottky diodes, 
% and a comprehensive model--hardware co-design \emph{specifically} for \textbf{ReRAM}-based PIM architectures.}
% By bridging the gap between neural network design and analog hardware constraints, our approach enables robust and energy-efficient image classification tailored for ultra-low-power and edge systems.
% %%%%% %%%%% %%%%%

%%%%%%%%%%%%%%%%%%%%%%%%%%%%%%%%%%%%%%%%%
%%%%%%%%%%%%%%%%%%%%%%%%%%%%%%%%%%%%%%%%%

 \section{Introduction}
\label{sec:intro}

The rapid proliferation of Internet of Things (IoT) devices in diverse and often hostile environments—such as wildlife monitoring~\cite{tuia2022perspectives,gobieski2019intelligence}, deep mine surveillance~\cite{bai2020real, ni2024detection}, and remote industrial automation~\cite{visconti2024machine, khalil2021deep}—demands \emph{efficient} and \emph{reliable} image classification solutions~\cite{RedEye2020, qiu2020resirca}. These applications characteristically operate under stringent energy constraints, where ultra-low power consumption, extended device lifetimes, and minimal maintenance are paramount~\cite{qiu2020resirca, gobieski2019intelligence}. For instance, wildlife monitoring sensors may need to run for years in remote habitats without battery replacements, while deep-mine surveillance nodes must withstand harsh conditions that render frequent maintenance impractical. To address these limitations, energy-harvesting technologies are increasingly employed, enabling devices to operate \emph{indefinitely} but often \emph{intermittently}~\cite{Lv2022, qiu2020resirca, ma2017incidental}. Under such conditions, traditional image classification approaches—usually reliant on power-hungry digital processing—fail to meet the rigorous energy constraints~\cite{RedEye2020, IMBNature}, thus demanding novel paradigms for \emph{energy-efficient computing}.

One promising line of research for achieving such efficiency targets is \emph{Processing-in-Memory} (PIM) architectures, particularly those that leverage \emph{Resistive RAM} (ReRAM) crossbars (xBars) for DNN-based inference~\cite{shafiee2016isaac}. Unlike standard digital designs, ReRAM xBars perform matrix-vector multiplications (MVMs) \emph{directly} in the analog domain, significantly reducing the energy overhead of data movement and digital processing~\cite{reramNature, chi2016prime, shafiee2016isaac, singh2020nebula}. Figure~\ref{fig:IntroDiagram} illustrates a typical image inference pipeline wherein data from image sensors are stored in memory (often eDRAM)~\cite{singh2020nebula, shafiee2016isaac, chi2016prime}, and a portion of the multiply-accumulate operations is offloaded to analog crossbars. Subsequently, activation functions and data-flow orchestration are conventionally managed by a microcontroller unit (MCU) or a dedicated control block. Further strengthening the prospects of ReRAM, its non-volatility and high endurance make it particularly suited for long-term reliability in low-power scenarios~\cite{chen2020reram, rizk2019demystifying}, and its compatibility with existing CMOS fabrication processes elevates its commercial viability~\cite{IMBNature}.

\begin{figure}[t]
    \centering
    \includegraphics[width=\linewidth]{figs/RichIntroDiagram.pdf}
    \vspace{-22pt}
    \caption{An example wildlife classification deployment---analog image sensor data is stored in digital memory, then processed in crossbar arrays in the mixed domain. The MCU, in the digital domain, performs activation functions. Every DNN layer undergoes expensive domain transfers.}
    \label{fig:IntroDiagram}
\end{figure}

Despite these notable advantages, significant hurdles remain in deploying ReRAM-based PIM systems for ultra low-power and intermittently powered settings. A primary challenge is the steep energy and latency overhead of analog-to-digital (A2D) and digital-to-analog (D2A) conversions~\cite{shafiee2016isaac, singh2020nebula}. As depicted in Figures~\ref{fig:IntroDiagram} and~\ref{fig:BaseArch}\textit{(I)}, while crossbar arrays compute MVM in the analog domain, crucial tasks such as activation functions, memory management, and control logic typically occur in the digital domain~\cite{shafiee2016isaac, singh2020nebula, chi2016prime, qiu2020resirca}. Modern DNNs rely heavily on nonlinear activations, particularly rectified linear units (ReLUs)~\cite{lecun1998gradient, hahnloser2000digital, hahnloser2000correction}, for boosting network performance~\cite{howard2017mobilenets}. Although the activation calculations themselves are often computationally light, the off-crossbar requirement for digital processing triggers frequent A2D and D2A conversions~\cite{shafiee2016isaac}, which can consume up to $82\%$ of the total power and occupy $31\%$ of the die area~\cite{shafiee2016isaac, chi2016prime, IMBNature}. Typical solutions offload these computations to external MCUs~\cite{singh2020nebula, qiu2020resirca} or specialized digital hardware~\cite{IMBNature}, but this offloading undercuts the central advantage of PIM—eliminating most data movement—and further inflates energy and latency. Notably, each off-crossbar activation step can add $15\%$ to $18\%$ latency overhead and approximately $5\%$ extra energy budget in ultra-low-power implementations~\cite{shafiee2016isaac, chi2016prime, IMBNature}.

Moreover, computational accuracy in analog crossbars suffers from IR drops and device-level variations inherent to large-scale arrays. Because crossbar computations hinge on Ohmic properties, voltage drop across rows and columns leads to uneven signal distribution, culminating in accuracy losses of up to $25\%$ in certain setups~\cite{IRDropDAC, IRDropTCAD, lee2024mitigating}. Existing mitigation techniques—ranging from current balancing to insertion of intermediate storage registers—amplify energy usage and degrade overall efficiency~\cite{crafton2022characterization, IRDropICCD}. Complicating matters, mainstream DNN architectures are typically designed with digital hardware assumptions, disregarding analog noise, device variability, and constrained precision in crossbars~\cite{lee2024mitigating, IRtrainingBL}. Such mismatch can introduce classification errors exceeding $15\%$~\cite{singh2020nebula} if models are not adapted to the underlying analog realities—an unacceptable performance drop for mission-critical or resource-starved applications.

To surmount these obstacles, we underscore the importance of jointly optimizing model architectures and hardware configurations. Few existing solutions fully account for how tile sizes, data representations, and analog hardware constraints interplay to affect energy efficiency and accuracy, particularly in ultra-low-power ReRAM designs~\cite{qiu2020resirca}. Although adaptive tiling has been explored, most lack comprehensive co-design of both the neural network and PIM architecture. Bridging this gap is paramount for extracting the full performance and energy gains that ReRAM crossbars can offer in edge deployments.

In this paper, we propose a novel framework that integrates \emph{hardware-aware neural network design} with \emph{energy-efficient activation strategies}, delivering a holistic solution for deploying ReRAM-based systems under tight energy budgets. Our approach confronts the dual constraints of analog non-idealities and constrained power budgets, achieving \textbf{up to 68\%} improvement in energy efficiency and \textbf{25\%} reduction in latency. These gains enable robust, continuous image classification for domains like wildlife monitoring and industrial automation, even under intermittent or harvested power. The \textbf{key contributions} of this paper include:

\begin{itemize}[leftmargin=*]
    \item \textbf{Analog Activation with Schottky Diodes:} We propose a novel analog activation function employing Schottky diodes, which feature low forward voltage drop, rapid switching speeds, and negligible leakage. This design obviates the need for off-chip or off-crossbar digital activations, cutting energy consumption and latency while preserving seamless compatibility with analog crossbar operations.
    
    \item \textbf{Hardware-Aware Training:} We build a training pipeline that explicitly incorporates real-world hardware non-idealities. By modeling Schottky diode dynamics and IR drop behavior in the activation function, we enable the DNN to adapt to analog-domain constraints. We further incorporate hardware noise profiles into the loss function, ensuring that trained models remain robust against voltage fluctuations and noise—pivotal in ultra low-power applications.
    
    \item \textbf{Model-Hardware Co-Design:} Our approach enforces a co-design paradigm wherein DNN architectures and the underlying ReRAM xBar hardware are optimized in tandem. We introduce a tile-based design to facilitate parallel analog MVMs across crossbar arrays, and exploit the potential to compute two consecutive network layers entirely in the analog domain. 
    %Furthermore, we apply \emph{tile dropout} techniques during training to bolster the model’s resilience against partial or noisy data, a common occurrence in energy-scarce environments.
    
    \item \textbf{Comprehensive Evaluation:} We conduct extensive validations to gauge the effectiveness of our framework against state-of-the-art references, observing up to \textbf{68\%} energy savings and \textbf{25\%} latency reductions, all while preserving high classification accuracy. Specifically, across standard benchmarks, our analog-aware models yield up to \textbf{1.63\%} higher accuracy compared to 4-bit quantized networks. Furthermore, on the NTLNP wildlife image dataset~\cite{ntlnpdataset, tan2022animal, ntlnprepo}, our method achieves an \textbf{88.1\%} accuracy using MicroNet, surpassing the 4-bit baseline by \textbf{4.9\%} and outperforming RedEye by \textbf{6.06\%}. We also extend our evaluation to modern transformer-based architectures, underscoring our activation function’s broad applicability.
\end{itemize}

To the best of our knowledge, this is the first comprehensive approach to integrate hardware-aware training, a Schottky diode-based analog activation circuit, and a model--hardware co-design specifically tailored to \textbf{ReRAM}-based PIM accelerators. By reconciling neural network design with the realities of analog crossbar hardware, our strategy achieves \emph{robust, energy-efficient} image classification suitable for ultra-low-power and edge deployments.


% %%%%%%%%%%%%%%%%%%%%%%%%%%%%%%%%%%%%%%%%%%%
% %%%%%%%%%%%%%%%%%%%%%%%%%%%%%%%%%%%%%%%%%%%
% \section{Introduction}
% \label{sec:intro}

% The rapid expansion of Internet of Things (IoT) devices in diverse, often harsh environments—such as wildlife monitoring~\cite{tuia2022perspectives,gobieski2019intelligence}, deep-mine surveillance~\cite{bai2020real, ni2024detection}, and remote industrial automation~\cite{visconti2024machine, khalil2021deep}—has created a pressing need for highly \emph{energy-efficient} and \emph{robust} image classification solutions~\cite{RedEye2020, qiu2020resirca}. These deployment scenarios typically demand ultra-low-power consumption and minimal maintenance over extended timeframes~\cite{gobieski2019intelligence, qiu2020resirca}, with devices often operating on tightly constrained power budgets or intermittent energy-harvesting schemes~\cite{Lv2022, qiu2020resirca, ma2017incidental}. For instance, wildlife monitoring sensors in remote forests must run continuously for years with limited battery replacements, and deep-mine surveillance systems may have to withstand extreme temperatures, physical obstructions, and hazardous conditions with infrequent recharges.

% In pursuit of these stringent requirements, \emph{Processing-in-Memory} (PIM) architectures based on \emph{Resistive RAM} (ReRAM) crossbars (xBars) have emerged as a transformative solution for deep neural network (DNN) inference~\cite{shafiee2016isaac, chi2016prime, singh2020nebula}. ReRAM xBars directly execute matrix-vector multiplications (MVMs) in the analog domain, mitigating energy overheads associated with data movement and memory accesses~\cite{reramNature, singh2020nebula}. Their non-volatile nature and compatibility with modern CMOS processes further bolster their suitability for embedded and edge AI~\cite{chen2020reram, rizk2019demystifying, IMBNature}. Figure~\ref{fig:IntroDiagram} depicts a typical ReRAM-based image inference pipeline in which analog operations (e.g., partial multiply-accumulate computations) occur inside crossbar arrays, while a microcontroller or specialized digital block handles activation functions and orchestrates the data flow.

% Despite these advantages, realizing \emph{fully analog inference} at ultra-low-power faces three critical challenges. First, frequent analog-to-digital (A2D) and digital-to-analog (D2A) conversions—required each time intermediate results or activations move between analog and digital domains—can consume up to $82\%$ of the overall power and $31\%$ of the area in PIM accelerators~\cite{shafiee2016isaac, chi2016prime, IMBNature}. Second, large-scale crossbar arrays suffer from IR drops and inherent hardware noise, triggering notable accuracy degradation if the network is not tailored to these analog non-idealities~\cite{IRDropDAC, IRDropTCAD, lee2024mitigating, IRDropICCD}. Third, many prior approaches lack a holistic co-design framework that \emph{simultaneously} optimizes both the DNN model and the underlying PIM hardware, missing opportunities to leverage tile-level parallelism, reduce conversion overheads, and incorporate real device parameters into training~\cite{qiu2020resirca, singh2020nebula}.

% In this paper, we tackle these gaps by introducing a comprehensive framework that unifies \emph{hardware-aware neural network training} with \emph{energy-efficient analog activation} in ReRAM xBars. Specifically, we replace the conventional digital off-chip activation with a Schottky-diode-based analog activation module, dramatically reducing cross-domain conversions. We further strengthen our \emph{co-design methodology} to account for hardware noise, IR drops, and intermittent power scenarios, and we experimentally validate our approach on both classic network architectures and emerging transformer-based designs. Our key contributions are as follows:

% \begin{itemize}[leftmargin=*]
%     \item \textbf{Analog Activation via Schottky Diodes:} We propose a novel analog activation circuit leveraging Schottky diodes for low forward voltage drop, rapid switching, and minimal leakage current. By embedding the non-linear activation directly in the analog domain, our design avoids repeated A2D/D2A conversions, substantially cutting energy and latency costs.
%     %
%     \item \textbf{Hardware-Aware Training:} We incorporate Schottky diode characteristics and crossbar-specific non-idealities (e.g., IR drop, device variability, and noise) into the training pipeline. The resulting networks are \emph{analog-aware}—they learn to maintain high classification accuracy under the voltage and signal distortions endemic to ReRAM crossbars, which is vital in ultra-low-power deployments.
%     %
%     \item \textbf{Model--Hardware Co-Design:} Beyond analog activations, we outline a tile-based crossbar architecture that processes two consecutive DNN layers fully in analog, reducing back-and-forth domain transfers. Techniques like \emph{tile dropout} bolster resilience to partial node failures or incomplete powering in energy-harvesting systems. This joint optimization of model topology and hardware resources closes the gap between analog capabilities and DNN requirements.
%     %
%     \item \textbf{Comprehensive Evaluation:} We extensively evaluate our framework on multiple benchmarks, including wildlife images~\cite{ntlnpdataset, tan2022animal, ntlnprepo} and transformer-based models. Our analog-aware networks achieve up to \textbf{68\%} energy savings and \textbf{25\%} latency reductions compared to state-of-the-art baselines, while maintaining—and in some cases improving—top-1 accuracy by up to \textbf{4.9\%} over a 4-bit quantized baseline in real-world scenarios.
% \end{itemize}

% \begin{figure}[t]
%     \centering
%     \includegraphics[width=\linewidth]{figs/RichIntroDiagram.pdf}
%     \caption{An example wildlife classification deployment. Analog image sensor data are stored in on-chip memory, partially processed via ReRAM crossbars, and then passed to an MCU for data orchestration. In typical designs, frequent cross-domain transfers for activation functions incur significant overheads. By embedding analog activations, our approach reduces these energy and latency bottlenecks.}
%     \label{fig:IntroDiagram}
% \end{figure}

% \noindent \textbf{Significance.} To the best of our knowledge, this is the first work to \emph{jointly} introduce a Schottky-diode-based analog activation design, a robust hardware-aware training flow, and a holistic model--hardware co-design specifically tailored for ReRAM-based PIM accelerators. By aligning the network’s structure with real-world device behavior, our solution not only maximizes energy efficiency but also maintains accuracy under power-scarce and noisy conditions, thus paving the way for next-generation ultra-low-power IoT deployments.

% %%%%%%%%%%%%%%%%%%%%%%%%%%%%%%%%%%%%%%%%%%%
% %%%%%%%%%%%%%%%%%%%%%%%%%%%%%%%%%%%%%%%%%%%

% %-------------------------------------------------------------------------------

% %-------------------------------------------------------------------------------
%\vspace{-4pt}
\section{\textbf{Background and Motivation}}
\label{sec:bg-rw}
%%%%%

\begin{figure*}[]
 \centering
    \subfloat[Standard DNN inference architectures: (I) PIM based, (II-III) Micro controller based]{\includegraphics[width=\linewidth]{figs/MotivationArchitecture.pdf}\label{fig:BaseArch}}%
    
    \subfloat[Understanding the trends, and details of the current deployments: (I -- IV) Understanding the breakdown of area and power for edge DNN deployment; (V) Higher power efficiency due to PIM; (VI-VII) Impact of hardware aware training on accuracy.]{\includegraphics[width=\linewidth]{NewResultFigs/combined_analysis_complete_final.pdf}\label{fig:loss}}%
    %\vspace{-8pt}
    \caption{An understanding of the current state of the art: (a) Different available architectures for deploying edge intelligence; (b) Contribution of the different moving parts of the edge ecosystem towards inference accuracy, and resource utilization.
    }
    \label{fig:Motivation}
\end{figure*}

% -----------------------------------------------------------------------------
% Section 2 – Background & Motivation
% -----------------------------------------------------------------------------
The appetite for on‑device intelligence continues to grow unabated: cameras that catalog endangered wildlife, vibration nodes that forecast industrial faults, and environmental beacons that harvest microwatts from ambient light are all expected to execute deep learning workloads where cloud connectivity is either intermittent or undesirable.  In such scenarios, the energy available for computation is typically measured in microjoules per inference, and any joule spent ferrying data between memory and logic is a joule that cannot be spared~\cite{qiu2020resirca, song2020drq, gobieski2019intelligence, maeng2018adaptive, IMBNature}.  Figure~\ref{fig:Motivation} juxtaposes three hardware datapaths that currently compete for these edge deployments and exposes the architectural friction points that motivate our work.

\noindent \textbf{The Baseline MCU Route:}  A traditional micro‑controller unit~\cite{msp_exp430fr5994} (\textit{Figure~\ref{fig:Motivation}(a,II)}) executes each multiply–accumulate (MAC) in the digital domain.  Even when the core is augmented with a low‑energy accelerator (LEA)~\cite{msp_exp430fr5994, TILEA} for vector arithmetic, the device must still marshal every feature map through on‑chip SRAM, dispatch DMA transfers, and consult a 12‑bit successive‑approximation ADC for sensor input.  The pie charts in \textit{Figure~\ref{fig:Motivation}(b,II–III)} quantify the result: more than three‑fifths of the total power is squandered on memory and I/O traffic, while the arithmetic engine that designers labor to optimize consumes barely one‑third.

\noindent \textbf{What an Analog PIM Can Offer:}  At the opposite extreme, a ReRAM crossbar array~\cite{qiu2020resirca, singh2020nebula, reramNature} (\textit{Figure~\ref{fig:Motivation}(a,I)}) stores weights as conductance and performs an entire matrix–vector multiplication in a single Ohmic step.  In principle, this in-situ analog arithmetic delivers more than two orders of magnitude better computational efficiency than a digital MCU pipeline; \textit{Figure~\ref{fig:Motivation}(b,V)} hints at this gap, with the crossbar  going >{150}{GOps/watt}~\cite{singh2020nebula} while the MCU stall below {5}{GOps/watt}.  Unfortunately, the same analysis also reveals an inconvenient truth: once the currents leave the array, they must be digitized so that a micro‑controller can apply a non‑linear activation and orchestrate the next layer.  The data converters that perform this bookkeeping account for more than eighty percent of the ReRAM power budget (\textit{Figure~\ref{fig:Motivation}(b,I)}), and they monopolize almost one‑third of the silicon area (\textit{Figure~\ref{fig:Motivation}(b,IV)}).  Every layer therefore suffers a round‑trip through an ADC, through the MCU, and back through a DAC before the next vector is launched, thus obliterating much of the intrinsic advantage promised by analog PIM~\cite{singh2020nebula, shafiee2016isaac, chi2016prime}.

\noindent \textbf{Why Existing Remedies Fall Short:}  Designers have responded along two axes.  Digital accelerators seek to overlap DMA transfers with MAC execution or compress activations in transit, yet the memory wall shown in \textit{Figure~\ref{fig:Motivation}(b,III)} remains stubborn.  Fully analog image sensor pipelines such as \textit{RedEye} move small convolution kernels into the focal plane, but they trade flexibility for leakage‑prone capacitors and struggle to scale beyond a handful of layers.  State-of-the-art ReRAM platforms like ISAAC~\cite{shafiee2016isaac}, PRIME~\cite{chi2016prime}, Nebula~\cite{singh2020nebula}, and ResiRCA~\cite{qiu2020resirca} push the arithmetic efficiency envelope, but all ultimately capitulate to the same mixed‑signal boundary: after {\em every} crossbar operation, the partial sums are digitized so that a rectified‑linear‑unit (ReLU) can be executed in software.

\noindent \textbf{Analog Noise Aware Training:} Accuracy in analog ReRAM crossbars can degrade significantly due to device variability, IR drop, and circuit-level noise. Without proper compensation, accumulated analog errors can lead to substantial accuracy losses. For example, when deploying MicroNet~\cite{li2021micronet} on CIFAR-10~\cite{cifar} (Refer Figure~\ref{fig:loss}(VI-VII)) using a 128$\times$128 ReRAM crossbar, the ideal FP32 accuracy of approximately 94\% is maintained in higher precisions (e.g., FP16 yielding around 93\%), while naive quantization to 4-bit (INT4) results in only 90\% accuracy. By contrast, incorporating analog-aware training—which explicitly models IR drop, transistor and diode non-linearities, and other process variations via quantization-aware training~\cite{qat} (QAT)—enables the network to recover much of the lost performance, with the 4-bit model achieving 91.63\% top-1 accuracy. 

This hardware-aware training approach not only adapts the weights and activations to the limited signal range and noise of the analog domain but also utilizes a realistic analog noise model to simulate the effects of process variations during the forward pass. The technique effectively reduces the accuracy gap between ideal digital computation and real-world analog behavior. In a similar vein, the \emph{LeCA}~\cite{ma2023leca} framework adopts a task-specific co-design strategy where an in-sensor autoencoder is jointly trained with the downstream classifier. This enables the system to learn compressive features that are robust to aggressive analog quantization and noise. For instance, LeCA achieves only a 0.97\% accuracy loss at a 4$\times$ compression ratio on ImageNet, and a 2.01\% loss at 8$\times$ compression, demonstrating that integrating analog non-idealities into the training loop substantially mitigates the degradation typically associated with low-bit analog implementations. 

Together, these results highlight the necessity of analog-aware co-training to preserve DNN accuracy under real-world hardware noise conditions, thereby ensuring energy-efficient deployment on analog processing-in-memory systems.


\noindent \textbf{An Opportunity at the Analog–Digital Seam:}  The evidence in Figure~\ref{fig:Motivation} therefore suggests a different approach.  If the dominant non‑linearity ReLU could be realized directly in the ``current domain'', and if the inevitable analog non‑idealities (IR drop, device mismatch, thermal drift) could be accounted for during training, then two consecutive neural network layers could be {\em fused} inside the crossbar fabric before a single low‑precision conversion.  Hence, eliminating just one ADC/DAC cycle out of every two slashes both energy and latency, and it does so {\em without}  forfeiting the programmability that a digital top‑level controller affords.

The remainder of this paper explores how close we can come to that ideal.  We begin by revisiting the circuit primitives available in standard processes and show that a modest Schottky‑diode network can {\em approximate} ReLU with a forward drop of only {{0.15}{V}}.  We then embed a model of that circuit, along with spatial IR‑drop maps and device‑level variability, into the training loop so that the network learns to live with the hardware it will ultimately inhabit.  Finally, we build a 64‑tile prototype around the  ISAAC‑class~\cite{shafiee2016isaac} crossbars and demonstrate, on contemporary CNNs and compact transformers, the energy–latency gains that Figure~\ref{fig:Motivation} foreshadows.
%%%%%

% %-------------------------------------------------------------------------------

% %-------------------------------------------------------------------------------
%\vspace{-4pt}
%\section{Optimizing Activation Function Computation}
\label{sec:Activation}
                         % %%%%% %%%%% %%%%%
% -----------------------------------------------------------------------------
% Section 3 – Analog Activation via Schottky‑Diode Network
% -----------------------------------------------------------------------------

\section{Analog Activation via Schottky Diodes}
\label{sec:analog-activation} 
Deep neural networks critically rely on nonlinear activation functions, such as the Rectified Linear Unit (ReLU), to learn complex representations. However, as established in Section~\ref{sec:bg-rw}, implementing ReLU purely in digital logic forces every analog output from the ReRAM crossbar to be digitized, evaluated in a microcontroller or accelerator, and then reconverted to analog for the next layer---an expensive process in terms of both power and area~\cite{singh2020nebula, shafiee2016isaac}. To eliminate these costly domain transfers, we propose to \emph{replace the digital ReLU with a diode-based analog activation} that can be placed directly at the output of each crossbar column.

\begin{figure}[]
\centering 
\includegraphics[width=\linewidth]{NewResultFigs/DiodeActivationCircuit2.pdf}
\caption{Implementation of ReLU-like activation function in the analog domain using diodes. We select a diode and resistor in series configuration (\circled{1}) for its simplicity and efficiency. We also include a transimpedance amplifier based driver for current to voltage domain crossing (\circled{2}).}   
\label{fig:ReLUDiode} 
\end{figure} 

\noindent
\textbf{Circuit Realization and Working Principle:} 
Figure~\ref{fig:ReLUDiode} illustrates three potential diode-based designs for an analog ReLU. After systematically assessing the forward voltage drop, leakage current, and circuit area, we adopt the single-Schottky-diode configuration because a Schottky diode offers a low forward drop ({$\sim$0.15\,V}) and fast switching. Here, a series resistor $R_s$ and an optional small damping resistor $R_d$ set the operating slope.

When the crossbar output voltage $V_i$ (converted from current by a transimpedance amplifier; see below) exceeds the diode's forward threshold $V_f$, the diode conducts and $V_o \approx V_i - V_f - I_D R_s$, closely emulating the positive branch of ReLU. Conversely, when $V_i \leq V_f$, the diode is reverse-biased and no current flows, causing $V_o \approx 0$. For a small-signal approximation, the diode current $I_D$ can be described by
$%\[
   I_D \;=\; I_S \Bigl(e^{\!\frac{V_D}{nV_T}} \,-\, 1\Bigr)\;,
$%\]
where $I_S$ is the reverse saturation current, $V_D$ is the diode drop, $n$ is the diode ideality factor, and $V_T$ is the thermal voltage (about {25\,mV} at room temperature). In normal forward operation ($V_D \gg nV_T$), $I_D \approx I_S \exp\bigl(\tfrac{V_D}{nV_T}\bigr)$ and thus even small voltage changes can trigger conduction. 

\noindent
\textit{Transimpedance Amplifier (TIA) as Driver:}
ReRAM crossbars naturally produce an output \emph{current} determined by $I_{\text{xbar}} = \sum_j G_{ij}\,V_j$. To feed the Schottky diode, we must first convert that current into a stable voltage domain. As shown conceptually in Figure~\ref{fig:ReLUDiode}, each crossbar column is connected to a simple TIA that maintains a low-impedance virtual node for the crossbar while generating a voltage $V_i$ proportional to the total column current.
%The diode-based activation circuit then operates around $V_i \approx \textcolor{red}{\text{user-specified range}}$, ensuring consistent threshold behavior even under variations in load or input size. 

\noindent
\textbf{Replacing ReLU with a Diode Activation Function:}
Although the circuit itself is governed by exponential diode laws, we find that a piecewise linear approximation greatly simplifies neural network training:
\begin{equation}
f_{\text{diode}}(x)\;=\;
\begin{cases}
0, & x \,\leq\, V_{\text{th}},\\[4pt]
\alpha\,\bigl(x - V_{\text{th}}\bigr), & x \,>\, V_{\text{th}},
\end{cases}
\label{eq:diode-activation}
\end{equation}
where $x$ is the TIA output, $V_{\text{th}} \approx V_f$ is the diode “turn-on” voltage, and $\alpha$ is a small slope factor set by $R_s$, $R_d$, and the diode's internal $I$--$V$ response. For $x \leq V_{\text{th}}$, $f_{\text{diode}}(x)=0$ mimics the ReLU’s zero output. Once $x$ exceeds $V_{\text{th}}$, the circuit conducts with a linear ramp of slope $\alpha$. In practice, $V_{\text{th}}$ can be \emph{slightly} higher than the ideal $V_f$ to account for conduction losses and series parasitics. To train a neural network that uses $f_{\text{diode}}(\cdot)$ instead of ReLU, we must compute its gradient during backpropagation, which is straingtformward given the equations.
%The derivative is: 
% \[
% f'_{\text{diode}}(x)
% \,=\,
% \begin{cases}
% 0,   & x \leq V_{\text{th}},\\
% \alpha, & x > V_{\text{th}}.
% \end{cases}
% \]
Thus, for each hidden/output neuron $i$ in layer $l$, the forward activation $a_i^{(l)} = f_{\text{diode}}\bigl(z_i^{(l)}\bigr)$ is straightforward; the backprop error $\delta_i^{(l)}$ multiplies $\alpha$ if $z_i^{(l)} > V_{\text{th}}$ and is zero otherwise. In effect, $f_{\text{diode}}(\cdot)$ behaves similarly to a “shifted” ReLU but with an analog realization.

\noindent
\textbf{Accommodating Hardware Variations:}
A crucial distinction from standard digital ReLU is that real diodes exhibit fluctuations in threshold voltage ($V_{\text{th}}$), slope $\alpha$, and leakage current, all of which are sensitive to temperature, process corners, and device aging~\cite{TIA, cmos1, cmos2}. We model these variations in training by sampling $(V_{\text{th}}, \alpha)$ from normal distributions centered at nominal diode parameters ({e.g.,\ $\mu_{V_{\text{th}}}=0.15\,\mathrm{V},\,\sigma=0.01\,\mathrm{V}$}), thereby exposing the network to a range of realistic hardware conditions. In each training minibatch, we randomly perturb these parameters for every neuron, forcing the learned weights to become robust against small mismatches. Specifically, if $V_{\text{th}}^{(l,i)}$ and $\alpha^{(l,i)}$ denote respectively the threshold and slope for neuron $i$ in layer $l$, we sample them once per minibatch from:
$%\[
V_{\text{th}}^{(l,i)} \sim \mathcal{N}\Bigl(\mu_{V_{\text{th}}},\,\sigma_{V_{\text{th}}}^2\Bigr),
\quad
\alpha^{(l,i)} \sim \mathcal{N}\Bigl(\mu_{\alpha},\,\sigma_{\alpha}^2\Bigr),
$%\]
where the network is \emph{not} learning these diode values directly but, rather, learning the weights and biases that can tolerate them. Such hardware-aware training helps maintain accuracy even when the environment deviates from nominal conditions at inference time. Additionally, the TIA can be susceptible to offset and gain errors; these are folded into the same variation model or handled as small input biases. More pronounced non-idealities, such as IR drop in large crossbars or dynamic noise injection, will be addressed in Section~\ref{sec:analogTraining}, where we show that an integrated approach—one that includes both the analog activation and the crossbar’s voltage/current limitations—leads to a more robust final model.

\noindent
\textbf{Enabling Consecutive Analog Layers:}
Once we can apply an activation directly in the analog voltage domain, it becomes possible to {\em chain} together \emph{two} crossbar layers before converting outputs to digital. Specifically, after the first crossbar’s MVM completes, we transform the currents to $V_i$ via TIA, apply the diode activation, and feed the resulting voltages directly into the next layer’s crossbar. Only after two layers do we quantize the partial sums with a low-bit ADC. This strategy cuts {\emph{one entire ADC/DAC cycle for every two layers}}, substantially reducing both energy and latency (detailed in Section~\ref{sec:eval}). To exploit this, however, the network must be trained in a manner that tolerates the small drift and overhead introduced by repeatedly staying in analog form, reinforcing the need for hardware-awareness in the optimization loop.

\noindent
\textbf{Key Takeaway:} Our diode-based circuit realizes a ReLU-like function entirely in the analog domain and offers a straightforward gradient for efficient backpropagation. By properly accounting for $V_{\text{th}}$ and $\alpha$ variations during training, we can ensure that the final deployed model remains robust despite the inherent variability of analog components. Next, we extend this hardware-awareness to the crossbar itself, modeling IR drop and noise sources so that the network learns to adapt to real-world analog imperfections.

%%%%%



%\input{src/04DiodeActivationTraining}
% %-------------------------------------------------------------------------------

% %-------------------------------------------------------------------------------
%\vspace{-4pt}
%\section{IR Drop}
%\section{Incorporating Analog Circuit Behaviors}
%into the Training Process}
\label{sec:analogTraining}
% %%%%% %%%%% %%%%%
\section{Analog-Aware Trainings}
\label{sec:analogTraining}
Deploying neural networks on analog ReRAM crossbars requires accounting for a range of circuit ``non-idealities" to ensure robust inference. In addition to device-level variations in the diodes (Section~\ref{sec:analog-activation}), the crossbar itself can suffer from IR drops, noise, and multi-bit cell variability. Furthermore, the transimpedance amplifier (TIA) that converts crossbar currents to voltages introduces its own offset and gain errors. In this section, we describe how all these analog factors are integrated into a ``hardware-aware" training procedure so that the network learns to mitigate them, preserving accuracy while enabling extended analog domain computations.

\subsection{Incorporating IR Drop and TIA Variations}
\textbf{IR drop in crossbars.} Large ReRAM arrays experience spatial voltage drops along the word and bit lines, causing cells at different locations to see unequal voltages and therefore produce inconsistent outputs~\cite{IRDropICCD, IRDropTCAD}. To approximate this, we treat the crossbar interconnect as having uniform resistances $R_W$ and $R_B$ per unit length on the word and bit lines, respectively. Each memristor at crosspoint $(i, j)$ has nominal resistance $R_{\text{mem}_{ij}}$, but the actual voltage it receives is reduced by
$%\[
\Delta V_{ij} = I_{\text{out},ij}\,\bigl(R_W^{(i)} + R_B^{(j)}\bigr),
$%\]
where $I_{\text{out},ij}$ is the local cell current. The net attenuation can be captured by a factor
$%\[
\gamma_{ij} \;=\; 1 \;-\; \frac{\Delta V_{ij}}{V_{\text{in}}},
$%\]
which indicates that only $\gamma_{ij}\,V_{\text{in}}$ is effectively applied across $R_{\text{mem}_{ij}}$. Although one could compute a unique $\gamma_{ij}$ for each cell, this can clearly become expensive for large arrays. Instead, we adopt a practical layer-wise scaling approach by letting each layer $l$ have a \emph{single} attenuation parameter $\gamma^{(l)}$ that represents an average or worst-case IR drop for that layer:
$%\[
a_i^{(l)} \;=\; \gamma^{(l)} \;\Bigl[a_i^{(l), 0}\Bigr],
$%\]
where $a_i^{(l), 0}$ is the neuron’s activation before the IR-drop adjustment. By allowing $\gamma^{(l)}$ to be learned or sampled, the network can develop internal redundancy to compensate for IR drops~\cite{he2019noise,lee2024mitigating}.

\noindent
\textbf{TIA Offset and Gain Errors:} After partial sums leave the crossbar, a TIA converts the current into a voltage suitable for our diode-based activation (Section~\ref{sec:analog-activation}). However, TIAs can exhibit an offset voltage $\delta_{V}$ and a gain deviation $\delta_{G}$ (i.e., $V_{\text{TIA}} = (1+\delta_{G})\,R_f\,I_{\text{xbar}} + \delta_{V}$). We treat these as random variables with small standard deviations around nominal designs~\cite{TIA}, and integrate them in the same manner as IR drops:
$%\[
V_{\text{TIA}} \leftarrow \bigl(1+\delta_{G}\bigr)\,V_{\text{TIA}} \;+\;\delta_{V}.
$%\]
In training, we sample $(\delta_{G}, \delta_{V})$ from known distributions
%(e.g., \textcolor{red}{$\pm 2\%$ gain error, \(\pm 5\,\text{mV}\) offset})
, allowing the model to remain robust despite these analog front-end imperfections.

\subsection{Noise Injection and Multi-Bit Variations}
Beyond IR drops and TIA inaccuracies, analog circuits face thermal noise, flicker noise, and programming uncertainty in multi-level ReRAM cells. We incorporate these by:
%\begin{itemize}
%\item 
\textbf{1. Additive noise injection:} For each crossbar output, we add a Gaussian $\mathcal{N}\bigl(0,\sigma^2\bigr)$ with $\sigma$ set to match a target SNR. If the signal power is $\mathcal{P}_{\text{sig}}$, we define $\text{SNR} = 20\,\log_{10}(\tfrac{\mathcal{P}_{\text{sig}}}{\mathcal{P}_{\text{noise}}})$ and pick $\sigma$ accordingly~\cite{puglisi2018random}.
%\item 
\textbf{2. Multi-bit cell variation:} Each $R_{\text{mem}_{ij}}$ is sampled from $\mathcal{N}\bigl(\bar{R}_{ij},\,(\Delta R)^2\bigr)$ to reflect manufacturing tolerances and cycle-to-cycle drifts~\cite{xu2022multi}.
%\end{itemize}
All of these perturbations are injected into the \emph{forward pass} during each mini-batch, ensuring that the network learns to adapt to them.

\subsection{Unified Loss and Hardware-Aware Backprop}
We unify the above non-idealities—IR drop via $\gamma^{(l)}$, diode threshold shifts $V_{\text{th}}^{(l)}$, slope deviations $\alpha^{(l)}$, and noise injections—under a single training loop. Concretely, we augment the standard classification loss $\mathcal{L}_{\text{cls}}$ with regularization terms that discourage $\gamma^{(l)}$, $\alpha^{(l)}$, or $V_{\text{th}}^{(l)}$ from drifting too far from nominal design values:
% \begin{equation}
% \label{eq:total_loss_proposed}
% \mathcal{L} \;=\; 
% \mathcal{L}_{\text{cls}} 
% \;+\;\lambda_{\gamma}\,\sum_{l}\bigl(\gamma^{(l)}-1\bigr)^2
% \;+\;\lambda_{\alpha}\,\sum_{l}\bigl(\alpha^{(l)}-\mu_{\alpha}\bigr)^2
% \;+\;\lambda_{V_{\text{th}}}\,\sum_{l}\bigl(V_{\text{th}}^{(l)}-\mu_{V_{\text{th}}}\bigr)^2.
% \end{equation}
\begin{equation}
\label{eq:total_loss_proposed}
\begin{aligned}
\mathcal{L} &= 
\mathcal{L}_{\text{cls}}
+ \lambda_{\gamma}\sum_{l}(\gamma^{(l)}-1)^2
+ \lambda_{\alpha}\sum_{l}(\alpha^{(l)}-\mu_{\alpha})^2 \\ 
&\quad{}+ \lambda_{V_{\text{th}}}\sum_{l}(V_{\text{th}}^{(l)}-\mu_{V_{\text{th}}})^2.
\end{aligned}
\end{equation}

Here, $\mu_{\alpha}$ and $\mu_{V_{\text{th}}}$ are nominal circuit-level parameters (e.g., $\mu_{V_{\text{th}}}\approx 0.15\,\text{V}$), and the regularization coefficients $(\lambda_{\gamma},\lambda_{\alpha},\lambda_{V_{\text{th}}})$ control how tightly the training process adheres to physically plausible settings. In forward propagation, each parameter is sampled from a normal distribution reflecting process corners; in backward propagation, we compute gradients w.r.t.\ the network weights (and optionally $\gamma^{(l)}$) to compensate for these variations~\cite{ma2023leca}.

\subsection{Architectural Implications}
Having modeled the diode activation and the crossbar’s IR drop, our system design allows two consecutive analog layers to be computed {\em without} an intermediate A/D conversion. In practice, this means:
%\begin{enumerate}
%\item 
\textbf{Crossbar 1} completes an MVM, the TIA drives each column into the diode activation, yielding an analog voltage vector.
%\item 
That voltage vector immediately feeds \textbf{Crossbar 2} for the second MVM + diode stage.
%\item 
Only \emph{then} do we digitize the partial sums with a {$k$-bit ADC} (e.g., 4-bit).
%\end{enumerate}
While this halves the frequency of costly ADC/DAC cycles, it also makes the second layer more sensitive to accumulated IR drops and TIA offset. Our training routine’s integrated modeling of $\gamma^{(l)}$ and $(\delta_G,\delta_V)$ ensures the second layer remains reliable under these extended analog conditions. 

Empirically, we find that two layers is a sweet spot: additional layers in analog may magnify IR drop and noise beyond acceptable limits, unless more buffer amplifiers are inserted~\cite{chi2016prime,singh2020nebula}. This can increase area and power, offsetting any gains from skipping ADC conversions. Hence, {\em restricting ourselves to pairs of analog layers strikes a balance between system complexity and energy efficiency.} 

%\subsection{Quantization-Aware Training}
\noindent\textbf{Quantization-Aware Training:}
Finally, after two analog layers, the partial sums are quantized with low precision. We incorporate this step directly into training—known as quantization-aware training (QAT)—by rounding the intermediate activations to {4-bit or 8-bit} during the forward pass (only after every second layer). This ensures that the model learns to tolerate not only analog circuit variations but also the discretization error from the limited-resolution ADC. Our implementation uses a straight-through gradient estimator for backprop through this rounding operation. Combining QAT with analog variation modeling allows the network to fully exploit the \emph{two-layer analog pipeline} while retaining high accuracy.

\noindent
\textbf{Extension to Digital or MCU-Based Systems:}
Although the core of our design exploits analog computation for two consecutive layers without intermediate digitization, many practical systems use a digital microcontroller or accelerator for control or additional layers. Our hardware-aware training naturally extends to these hybrid pipelines by including appropriate quantization steps in the forward pass. For example, if an MCU enforces $n$-bit fixed-point arithmetic, we can quantize the outputs of certain layers to $n$ bits during training. Similarly, if sensor data arrive with a certain signal-to-noise range, we can inject that variation at the input to ensure the model accounts for real-world sampling imprecision. In this way, the same framework that corrects IR drop and diode shifts can also compensate for digital truncation or sensor-level noise. The network effectively learns a unifying error budget, distributing redundancy across weights so that neither analog nor digital imperfections collapse inference accuracy.

% \noindent
% \textbf{Key Takeaway:} By jointly modeling IR drop, TIA imperfections, diode parameter shifts, and noise during training, we equip the network to overcome the pitfalls of analog computing. Section~\ref{sec:eval} will demonstrate that this hardware-aware approach can sustain accuracy under realistic crossbar conditions, enabling efficient two-layer analog execution with minimal performance loss.
%%%%% %%%%% %%%%%


% %-------------------------------------------------------------------------------

% %-------------------------------------------------------------------------------
%\vspace{-4pt}
%\section{Unified Design}
% \section{System Integration and Overall Architecture}

% %%%%% %%%%% %%%%%
\section{System Integration and Overall Architecture}
%\label{sec:system-arch} 
\label{sec:SystemIntegration}
This section consolidates the previously introduced components --ReRAM crossbars with diode-based analog activations, IR-drop-aware training, and a tile-based layout-- into a comprehensive accelerator system. The primary objective is to house a complete 4-bit quantized DNN on-chip and execute two consecutive layers in the analog domain before a single digitization step. By doing so, the design mitigates the overheads and device wear stemming from frequent ReRAM writes or excessive ADC use.

\noindent
\textbf{Overall Goals and Constraints:}
The ReRAM arrays are configured so that each DNN weight is programmed \emph{once} during system initialization, eliminating reprogramming during inference. This choice addresses both ReRAM endurance limitations and the high energy costs of repeated writes. By grouping two network layers into a single analog processing flow, the system halves the number of analog-to-digital conversions compared to a baseline that digitizes after every layer. To maintain accuracy under IR drop and noise as partial sums propagate across two analog layers, our hardware-aware training strategy (Section~\ref{sec:analogTraining}) incorporates these circuit non-idealities into the training loop.

\begin{figure}[]
\centering 
\includegraphics[width=\linewidth]{NewResultFigs/FullDesign.pdf}
\caption{Overall design of the proposed system: Left shows the high-level hardware and right shows the intricate hardware software co-design and training.}   
\vspace{-10pt}
\label{fig:overall} 
\end{figure} 

\noindent
\textbf{Crossbar Array and Tile Structure:}
Moderate crossbar dimensions, such as $64\times64$, offer a practical balance between chip area efficiency and increasingly significant IR drops in larger arrays. Each cell encodes a 4-bit weight via multi-level or bit-sliced programming. To accommodate the entire 4-bit DNN, the system instantiates enough crossbars to store all network parameters. For example, a three-million-parameter MobileNetV3~\cite{mobilenetv3} at four bits per weight (roughly 1.7\,MB) maps to a few hundred such arrays. Groups of these crossbars form a ``tile," complete with local MeF-RAM~\cite{angizi2021mef, sanjeet2024mefet, najafi2024hybrid, morsali2023design} for activation or partial-sum storage, peripheral circuits (DACs, ADCs, TIAs, diode activations), and a tile-level controller. This organization allows each tile to perform multiply-accumulate operations and store intermediate results. Because MeF-RAM maintains data integrity under power loss, the design is well suited to energy-harvesting and intermittent-computing scenarios.

\noindent
\textbf{Two-Layer Analog Execution and Scheduling:}
A defining feature of the accelerator is its ability to perform a pair of layers entirely in the analog domain. Conceptually, an input vector is fetched from the tile’s MeF-RAM and driven onto the crossbar rows for the layer $(2k+1)$. The resulting column currents pass through transimpedance amplifiers and diode-based activations, producing partial sums that remain analog. Rather than converting these partial sums to digital, a small analog switch network redirects them to the row inputs of the next crossbar, which stores the layer $(2k+2)$. Once the second layer completes its analog multiply-accumulate and diode activation, a 4-bit ADC captures the final outputs. In deeper architectures, this pattern repeats, such that partial sums are digitized only after every second layer. The static mapping of weights—programmed once at initialization—ensures each layer resides on a fixed region of crossbars with {\em no} in-field reconfiguration, thus avoiding further wear on the ReRAM cells.

\noindent
\textbf{Tile Microarchitecture:}
Each tile maintains on-chip MeF-RAM as a buffer for activations, partial sums, or per-layer constants. Its crossbars employ TIAs at every column output, followed by the diode activation stage. The switch fabric that routes analog outputs from one crossbar to the next enables back-to-back analog layers without intermediate digital storage. A tile-level controller orchestrates the sequence of operations: it triggers the row-by-row data input, coordinates the transimpedance and diode-activation phases, and decides when to invoke the ADC after the second layer. Because ReRAM devices are programmed only once, there is no frequent write overhead, which safeguards their endurance rating. Additionally, the non-volatility of MeF-RAM permits checkpointing if power is lost, an important consideration for resource-constrained or energy-harvesting applications.

\noindent
\textbf{Large DNN Mapping Without Reprogramming:}
To scale up, the architecture simply increases the total number of crossbars until every layer of the network fits on-chip. Each layer or pair of layers is assigned to specific crossbars so that partial sums move locally, in analog form, within the same tile. Once two layers are completed, the partial sums are quantized and passed digitally to the tile that holds the subsequent pair of layers. This scheme preserves all network weights in the crossbars throughout inference and avoids writing to them again. Moderate crossbar dimensions also help us control the impact of IR drop, and any residual mismatch is managed through the IR-drop modeling embedded in our hardware-aware training pipeline.

\noindent
\textbf{System-Level Control and Execution Flow:}
An on-chip microcontroller (or equivalent global controller) initiates each inference, directing the tiles to load input vectors from MeF-RAM, run the analog MVM and diode activations, then forward the partial sums in analog to the next layer or invoke the 4-bit ADC if the second layer in the pair has just completed. This process continues until the final layer is reached, with {\em no} intermediate weight reprogramming. Post-processing tasks such as Softmax may be performed by the same controller or a simple digital block. The overall result is an accelerator pipeline that moves seamlessly between analog and low-precision digital domains.

\noindent
\textbf{Key Architectural Advantages:}
By restricting ReRAM writes to a single load phase at initialization, our proposed design avoids the endurance-related drawbacks that commonly challenge memristive accelerators. Grouping layers in pairs also substantially reduces ADC overhead. The embedded hardware-awareness prevents accuracy degradation in the presence of IR drop, noise, and diode I--V non-idealities, while MeF-RAM’s non-volatility delivers resiliency against power interruptions. Storing the entire DNN on-chip obviates the need for frequent memory transfers, localizing most of the traffic to the tile-level movement of activations and partial sums. Compared to conventional architectures that digitize after every layer or rely on reconfiguring crossbars mid-inference, our system operates with far fewer domain conversions and minimal ReRAM stress, offering a robust and energy-efficient approach for deploying models like MobileNetV3 in edge scenarios.

\noindent
\textbf{Extension to MCUs, Conventional Accelerators, and Sensor Noise:}
Although our design is centered on ReRAM crossbars with diode-based activations, the underlying hardware-awareness techniques—especially those modeling IR drops, diode variations, and multi-bit noise—can be transferred to conventional MCUs, digital accelerators, or real-world sensors. In a hybrid architecture where an MCU implements certain fixed-point or floating-point layers, the same variation models that capture analog errors can be repurposed to account for finite-precision accumulators or offset and gain drift in digital pipelines. Similarly, sensor noise and sampling artifacts are readily injected at the input stage by adding perturbation terms in the training loop, enabling the network to accommodate real-world signal degradation just as it does IR drops or TIA offsets. Once partial sums exit the analog crossbars, they can be digitized and passed to any off-the-shelf processing unit for subsequent layer computations, activation functions, or classification. This approach thus {\em unifies} analog crossbar execution, digital acceleration, and noisy sensor inputs under a single ``hardware-aware" training framework, preserving the efficiency and robustness benefits of our ReRAM-based design even when integrated into mainstream MCUs or commercial accelerators. 
% %%%%% %%%%% %%%%%

% %-------------------------------------------------------------------------------

% %-------------------------------------------------------------------------------
%\vspace{-4pt}
\section{Implementation and Evaluation}
\label{sec:eval}
%%%%% %%%%% %%%%%
\begin{figure*}[]
 \centering
    \subfloat[Accuracy]{\includegraphics[width=0.32\linewidth]{figs/acc.pdf}\label{fig:accuracy}}%
    \hfill
    \subfloat[Loss]{\includegraphics[width=0.32\linewidth]{figs/loss.pdf}\label{fig:loss}}%
    \hfill
    \subfloat[Gamma Regularization]{\includegraphics[width=0.32\linewidth]{figs/reg_gamma.pdf}\label{fig:gamma}}
    %\vspace{-8pt}
    \caption{Update trends for accuracy, loss, and $\gamma$-regularization during training. 
    These examples illustrate how the network adapts to IR drop and diode variations over epochs.
    }
    \label{fig:software}
\end{figure*}
In this section, we provide a comprehensive evaluation of our proposed hardware-software co-design approach. 
Section~\ref{sec:evalSetup} details the experimental setup, including baseline comparisons, model quantization, and measurement of latency. 
Sections~\ref{sec:evalImplementationHW} and~\ref{sec:evalImplementationSW} describe the hardware and software implementations, respectively, 
while Section~\ref{sec:evalBenchmarking} presents results on widely used benchmarks. 
Finally, Section~\ref{sec:evalCaseStudy} highlights a low-power wildlife monitoring scenario.
Finally, Section~\ref{sec:transformerEval} shows how the analog activation perform in modern langue models based on transformers.
\subsection{Experimental Setup and Methodology}
\label{sec:evalSetup}
The experiments target five neural networks trained on the TinyImageNet~\cite{tiny} dataset: MobileNetV3-Small~\cite{mobilenetv3}, EfficientNet-Lite0~\cite{Efficientnet}, MicroNet~(M0)~\cite{micronet}, 
MCUNet~\cite{mcunet}, and MobileViT-XXS~\cite{mobilevit}. Each is evaluated in Q4 quantization (our primary focus) as well as Q8 and Q16 references. 
We use Q4 because it reduces memory footprint and ADC/DAC overheads, which are especially critical for ultra-low-power or battery-driven systems~\cite{qiu2020resirca, singh2020nebula}, 
yet we also provide higher-precision results to illustrate the accuracy tradeoffs. 
When reporting noise robustness, for example in Figs.~\ref{fig:noise_impact} and~\ref{fig:classification_accuracy_snr_case_Study}, 
we define $\text{SNR (dB)} = 20 \log_{10}\bigl(\tfrac{P_{\text{signal}}}{P_{\text{noise}}}\bigr)$ and inject Gaussian noise at the output of each analog crossbar step; 
at 10\,dB SNR, the noise variance is scaled to ensure the signal power is ten times the noise power, simulating the thermal and flicker noise common in analog circuits~\cite{singh2020nebula}. 
For smaller networks such as MicroNet~(M0) or MCUNet, the entire model can be stored in our ReRAM-based design without reloading. 
For deeper architectures such as MobileNetV3-Small, EfficientNet-Lite0, or MobileViT-XXS, each layer's weights are fetched from local MeF-RAM or off-chip storage into the 64$\times$64 crossbar tiles, the analog operations are performed, and partial outputs are stored for the subsequent layer. These reloading overheads are included in the total power and latency estimates in Table~\ref{tab:DesignComponents} and Section~\ref{sec:evalBenchmarking}.

% \begin{table}[!h]
% \centering
% \resizebox{\linewidth}{!}{%
% \begin{tabular}{|clcc|}
% \hline
% \multicolumn{4}{|c|}{\textbf{Tile Processing Unit}}                                                                                                                  \\ \hline
% \multicolumn{1}{|l|}{\textbf{Component}}                   & \multicolumn{1}{l|}{\textbf{Parameter}} & \multicolumn{1}{c|}{\textbf{Specification}}     & \textbf{Power}          \\ \hline
% \multicolumn{1}{|c|}{\multirow{3}{*}{\textbf{MeF-RAM}}}      & \multicolumn{1}{l|}{Size}           & \multicolumn{1}{c|}{32\,kB}             & \multirow{3}{*}{8.5\,mW} \\ \cline{2-3}
% \multicolumn{1}{|c|}{}                                     & \multicolumn{1}{l|}{Banks}          & \multicolumn{1}{c|}{2}                 &                         \\ \cline{2-3}
% \multicolumn{1}{|c|}{}                                     & \multicolumn{1}{l|}{Bus Width}      & \multicolumn{1}{c|}{128\,bits}              &                         \\ \hline
% \multicolumn{1}{|c|}{\multirow{3}{*}{\textbf{Activation}}} & \multicolumn{1}{l|}{ReLU}           & \multicolumn{1}{c|}{Diode-based} & 1.37\,\(\mu\)W                  \\ \cline{2-3} \cline{4-4} 
% \multicolumn{1}{|c|}{}                                     & \multicolumn{1}{l|}{Sigmoid (LUT)}  & \multicolumn{1}{c|}{Lookup Table}                  & 0.2\,mW                   \\ \cline{2-3} \cline{4-4} 
% \multicolumn{1}{|c|}{}                                     & \multicolumn{1}{l|}{OR Gate}             & \multicolumn{1}{c|}{Logic Operation}                  & 250\,\(\mu\)W                   \\ \hline
% \multicolumn{4}{|c|}{}                                                                                                                                              \\ \hline
% \multicolumn{4}{|c|}{\textbf{Matrix-Vector Multiplication (MVM) Details (Total 64 Tiles)}}                                                                                                         \\ \hline
% \multicolumn{1}{|c|}{\multirow{3}{*}{\textbf{ADC}}} & \multicolumn{1}{l|}{Resolution} & \multicolumn{1}{c|}{4 bits} & \multirow{3}{*}{2.3\,mW} \\ \cline{2-3}
% \multicolumn{1}{|c|}{}                                     & \multicolumn{1}{l|}{Frequency}      & \multicolumn{1}{c|}{400\,MSps}          &                         \\ \cline{2-3}
% \multicolumn{1}{|c|}{}                                     & \multicolumn{1}{l|}{Number}         & \multicolumn{1}{c|}{4}                 &                         \\ \hline
% \multicolumn{1}{|c|}{\multirow{2}{*}{\textbf{DAC}}} & \multicolumn{1}{l|}{Resolution} & \multicolumn{1}{c|}{1 bit}  & \multirow{2}{*}{500\,\(\mu\)W} \\ \cline{2-3}
% \multicolumn{1}{|c|}{}                                     & \multicolumn{1}{l|}{Number}         & \multicolumn{1}{c|}{128 (4$\times$32)}              &                         \\ \hline
% \multicolumn{1}{|c|}{\multirow{3}{*}{\textbf{ReRAM xBar}}}       & \multicolumn{1}{l|}{Number}         & \multicolumn{1}{c|}{32}                & \multirow{3}{*}{1.34\,mW} \\ \cline{2-3}
% \multicolumn{1}{|c|}{}                                     & \multicolumn{1}{l|}{Size}           & \multicolumn{1}{c|}{32$\times$32}             &                         \\ \cline{2-3}
% \multicolumn{1}{|c|}{}                                     & \multicolumn{1}{l|}{Bits per Cell}  & \multicolumn{1}{c|}{4}                 &                         \\ \hline
% \end{tabular}%
% }
% \caption{Component specifications for the chip design.\vspace{-20pt}}
% \label{tab:DesignComponents}
% \end{table}

\begin{table}[!h]
\centering
\resizebox{\linewidth}{!}{%
\begin{tabular}{|clcc|}
\hline
\multicolumn{4}{|c|}{\textbf{Tile Processing Unit (38 Tiles Total)}}                                                                                                                  \\ \hline
\multicolumn{1}{|l|}{\textbf{Component}}                   & \multicolumn{1}{l|}{\textbf{Parameter}} & \multicolumn{1}{c|}{\textbf{Specification}}     & \textbf{Power}          \\ \hline
\multicolumn{1}{|c|}{\multirow{4}{*}{\textbf{MeF-RAM}}}      & \multicolumn{1}{l|}{Size}           & \multicolumn{1}{c|}{32\,kB per tile, 1.216\,MB total}             & \multirow{4}{*}{8.5\,mW} \\ \cline{2-3}
\multicolumn{1}{|c|}{}                                     & \multicolumn{1}{l|}{Banks}          & \multicolumn{1}{c|}{2}                 &                         \\ \cline{2-3}
\multicolumn{1}{|c|}{}                                     & \multicolumn{1}{l|}{Bus Width}      & \multicolumn{1}{c|}{128\,bits}              &                         \\ \cline{2-3}
\multicolumn{1}{|c|}{}                                     & \multicolumn{1}{l|}{Usage}      & \multicolumn{1}{c|}{Buffering activations, norms}              &                         \\ \hline
\multicolumn{1}{|c|}{\multirow{4}{*}{\textbf{Activation}}} & \multicolumn{1}{l|}{ReLU}           & \multicolumn{1}{c|}{Diode-based} & 1.37\,\(\mu\)W per channel                  \\ \cline{2-3} \cline{4-4} 
\multicolumn{1}{|c|}{}                                     & \multicolumn{1}{l|}{Channels}  & \multicolumn{1}{c|}{64 per crossbar}                  & 87.7\,\(\mu\)W per crossbar                   \\ \cline{2-3} \cline{4-4} 
\multicolumn{1}{|c|}{}                                     & \multicolumn{1}{l|}{Sigmoid (LUT)}  & \multicolumn{1}{c|}{Lookup Table}                  & 0.2\,mW                   \\ \cline{2-3} \cline{4-4} 
\multicolumn{1}{|c|}{}                                     & \multicolumn{1}{l|}{OR Gate}             & \multicolumn{1}{c|}{Logic Operation}                  & 250\,\(\mu\)W                   \\ \hline
\multicolumn{1}{|c|}{\multirow{2}{*}{\textbf{Controller/MCU}}} & \multicolumn{1}{l|}{Function}           & \multicolumn{1}{c|}{Softmax, layer norm, scheduling} & \multirow{2}{*}{5-10\,mW} \\ \cline{2-3}
\multicolumn{1}{|c|}{}                                     & \multicolumn{1}{l|}{Operation}          & \multicolumn{1}{c|}{4-bit or higher precision}                 &                         \\ \hline
\multicolumn{4}{|c|}{}                                                                                                                                              \\ \hline
\multicolumn{4}{|c|}{\textbf{Matrix-Vector Multiplication (MVM) Details}}                                                                                                         \\ \hline
\multicolumn{1}{|c|}{\multirow{4}{*}{\textbf{ADC}}} & \multicolumn{1}{l|}{Resolution} & \multicolumn{1}{c|}{4-bit SAR ADC} & \multirow{3}{*}{1.2\,mW at 200\,MSps} \\ \cline{2-3}
\multicolumn{1}{|c|}{}                                     & \multicolumn{1}{l|}{Frequency}      & \multicolumn{1}{c|}{200\,MSps (time-shared)}          &                         \\ \cline{2-3}
\multicolumn{1}{|c|}{}                                     & \multicolumn{1}{l|}{Per Tile}         & \multicolumn{1}{c|}{4 ADCs (152 total)}                 &                         \\ \cline{2-4}
\multicolumn{1}{|c|}{}                                     & \multicolumn{1}{l|}{Active}         & \multicolumn{1}{c|}{4-12 ADCs per layer}                 & 9.6\,mW (8 ADCs active)                         \\ \hline
\multicolumn{1}{|c|}{\multirow{3}{*}{\textbf{DAC}}} & \multicolumn{1}{l|}{Resolution} & \multicolumn{1}{c|}{1-bit (bit-serial inputs)}  & \multirow{2}{*}{0.5\,mW per driver} \\ \cline{2-3}
\multicolumn{1}{|c|}{}                                     & \multicolumn{1}{l|}{Number}         & \multicolumn{1}{c|}{64 drivers per active crossbar}              &                         \\ \cline{2-4}
\multicolumn{1}{|c|}{}                                     & \multicolumn{1}{l|}{Total Power}         & \multicolumn{1}{c|}{When all drivers active}              & 32\,mW (typical: partial usage)                         \\ \hline
\multicolumn{1}{|c|}{\multirow{5}{*}{\textbf{ReRAM xBar}}}       & \multicolumn{1}{l|}{Size}           & \multicolumn{1}{c|}{64$\times$64}             & \multirow{3}{*}{1.34\,mW per crossbar} \\ \cline{2-3}
\multicolumn{1}{|c|}{}                                     & \multicolumn{1}{l|}{Bits per Cell}  & \multicolumn{1}{c|}{4-bit (multi-level)}                 &                         \\ \cline{2-3}
\multicolumn{1}{|c|}{}                                     & \multicolumn{1}{l|}{Per Tile}         & \multicolumn{1}{c|}{32 crossbars}                &                         \\ \cline{2-4}
\multicolumn{1}{|c|}{}                                     & \multicolumn{1}{l|}{Total}         & \multicolumn{1}{c|}{1,220 crossbars (5M weights)}                & \multirow{2}{*}{85\,mW (64 crossbars active)}                         \\ \cline{2-3}
\multicolumn{1}{|c|}{}                                     & \multicolumn{1}{l|}{Usage}         & \multicolumn{1}{c|}{8-12 tiles active concurrently}                &                         \\ \hline
\multicolumn{1}{|c|}{\textbf{Total Power}}       & \multicolumn{1}{l|}{Full Inference}         & \multicolumn{1}{c|}{All tiles sequentially active}                & 150-300\,mW average                         \\ \hline
\end{tabular}%
}
\caption{Component specifications for the chip design. Power estimates dominated by ADCs and crossbars during MVM operations, with peaks managed through scheduling.}
\label{tab:DesignComponents}
\end{table}

\begin{figure}[]
  \centering
  \includegraphics[width=0.85\linewidth]{NewResultFigs/network_latency_comparison.pdf}
  \vspace{-8pt}
  \caption{Latency breakdown and savings for four representative networks. 
           Each group compares the average baseline per-layer overhead (left bar) with our proposed two-layer analog pipeline (right bar), showing consistent latency savings. On an average, across all layers in all the networks analyzed, by skipping ADC/DAC and MCU activation on alternate layers, we save $\approx 25\%$ of latency.
           \vspace{-8pt}
           }
  \label{fig:pipelineLatency}
\end{figure}

\subsection{Hardware Implementation}
\label{sec:evalImplementationHW}
The hardware design aims to optimize energy efficiency under tight power budgets. As summarized in Table~\ref{tab:DesignComponents}, 
each tile consists of MeF-RAM for buffering, a set of 64$\times$64 ReRAM crossbars, 4-bit SAR ADCs, 1-bit DAC drivers, diode-based activation circuits, and a lightweight MCU controller. 
Figure~\ref{fig:software} illustrates how training hyperparameters (accuracy, loss, and $\gamma$ regularization) evolve when the network is exposed to non-ideal analog conditions. 
The tile-level MeF-RAM is chosen for its nonvolatility and lower write energy compared to spin-based alternatives~\cite{angizi2021mef, sanjeet2024mefet, najafi2024hybrid, morsali2023design, usas}, 
which is beneficial for intermittent power scenarios; the slight increase in memory area is offset by its high read/write efficiency. 
Each ReRAM crossbar performs in-situ analog MAC operations on 4-bit weights, leveraging the inherent parallelism of the memristor array. 
The Schottky diodes enable ReLU-like activation directly in the analog voltage domain, reducing frequent analog-to-digital conversions. 
In Cadence Virtuoso~\cite{cadence_virtuoso} and Spectre~\cite{cadence_spectre} simulations (with a 65\,nm CMOS node~\cite{torki_65nm_pdk}), the diode exhibits a barrier height near 0.34\,eV, 
with careful tuning of doping and contact area achieving a forward drop around 0.15\,V and a leakage current on the order of 5\,nA, 
thus enabling fast, low-voltage switching. 
The 4-bit SAR ADCs operate at up to 400\,MSps with approximately 2.5\,ns conversion delay, while the 1-bit input DACs, being simpler, contribute negligible delay~\cite{shafiee2016isaac, chi2016prime}. 
This system-level integration allows two consecutive layers to be computed in analog, saving one ADC/DAC cycle for every pair of layers. 
After extensive IR-drop and diode-drop modeling in software, we find that processing more than two consecutive layers often necessitates additional analog buffering or an amplification stage to maintain accuracy. 
By limiting to two analog layers, we achieve an average latency reduction of around 25\% across all tested models, 
as shown in Figure~\ref{fig:pipelineLatency}, where the bars highlight the skip of MCU intervention and data converter overhead on alternating layers. 
Table~\ref{tab:DesignComponents} also lists the total power budget of 150--300\,mW; 
the dominant contributors are the crossbars and ADCs during simultaneous tile activity, 
though scheduling can mitigate peak loads. 
In scenarios where large models must be swapped in, weight programming consumes on the order of hundreds of picojoules per cell~\cite{IMBNature}, 
but these costs are typically amortized across many inferences, especially in deployment where the model remains static for extended periods.

\begin{figure*}[]
\centering
\begin{subfigure}[t]{0.49\linewidth}
    \centering
    \includegraphics[width=\linewidth, height=1in]{NewResultFigs/model_accuracy_final.pdf}
    \caption{Tiny ImageNet Classification accuracy comparison.}
    \label{fig:accuracy_comparison}
\end{subfigure}%
\hfill
\begin{subfigure}[t]{0.49\linewidth}
    \centering
    \includegraphics[width=\linewidth, height=1in]{NewResultFigs/power_efficiency_comparison.pdf}
    \caption{Power efficiency comparison (GOps/W) across different models.}
    \label{fig:power_efficiency}
\end{subfigure}%
\vskip\baselineskip
\begin{subfigure}[t]{0.49\linewidth}
    \centering
    \includegraphics[width=\linewidth, height=1in]{NewResultFigs/crossbar_utilization_comparison.pdf}
    \caption{xBar utilization rate comparison (\%).}
    \label{fig:xbar_utilization}
\end{subfigure}%
\hfill
\begin{subfigure}[t]{0.49\linewidth}
    \centering
    \includegraphics[width=\linewidth, height=1in]{NewResultFigs/accuracy_vs_noise_trend.pdf}
    \caption{{Noise vs Accuracy at different SNR for MicroNet.}}
    \label{fig:noise_impact}
\end{subfigure}
\vspace{-8pt}
\caption{Evaluations on publicly-available models, datasets, and benchmarks.}
\label{fig:combined_benchmarks}
\end{figure*}

\vspace{-16pt}
\subsection{Software Implementation}
\label{sec:evalImplementationSW}
Our software stack focuses on integrating the diode-based activation function and IR drop modeling into the training process. We use the PyTorch~\cite{Pytorch} framework 
for dynamic computation graphs and flexible custom layers.

\noindent\textbf{Implementation of Diode-Based Activation:} 
We implement a custom PyTorch module for the diode-based ReLU. 
In the forward pass, the piecewise linear function (Equation~\ref{eq:diode-activation}) is evaluated, 
where \(V_{\text{th}}\) and \(\alpha\) are sampled from normal distributions each iteration, simulating hardware variability. 
We subclass \texttt{torch.autograd.Function}~\cite{torch_autograd_function} to ensure correct autodiff flow. Figure~\ref{fig:loss} shows an example training run with these custom modules.

\noindent\textbf{Incorporation of IR Drop Factors:} 
Layer-wise attenuation \(\gamma^{(l)}\) modifies activations post-diode. Either pre-characterized IR-drop tables or random sampling can be used to handle manufacturing or environmental variability. Figure~\ref{fig:gamma} illustrates how \(\gamma^{(l)}\) converges during training.

\noindent\textbf{Quantization-Aware Training (QAT):} 
{
Whenever two layers are computed in analog consecutively, 
we apply a 4-bit quantization step at the output of the second layer to simulate the real ADC rounding. This ensures that our model is robust to the $\pm \tfrac{1}{2}\Delta$ quantization error introduced in hardware.}

\noindent\textbf{Code and Modifications:} 
We make minimal changes to the PyTorch core libraries, mainly extending \texttt{ toch.autograd.Function} for the custom activation and IR drop modules. Our entire implementation (data loaders, model definitions, training loops, evaluation) is about 2,500 lines of Python code, ensuring maintainability and compatibility.

\subsection{Evaluation on Public Benchmarks}
\label{sec:evalBenchmarking}
We evaluate our approach on five networks trained at 4-bit precision (Q4): MobileNetV3-Small~\cite{mobilenetv3}, EfficientNet-Lite0~\cite{Efficientnet}, MicroNet~(M0)~\cite{micronet}, MCUNet~\cite{mcunet}, and MobileViT-XXS~\cite{mobilevit}. 
For reference, we also include Q8 or Q4 baselines where available. 
Our experiments compare both accuracy and hardware metrics to several ReRAM-based or analog PIM platforms—ISAAC~\cite{shafiee2016isaac}, ResiRCA~\cite{qiu2020resirca}, Nebula~\cite{singh2020nebula}, HARDSEA~\cite{hardsea}, 
and RedEye~\cite{RedEye2020}—focusing on power efficiency (Figure~\ref{fig:power_efficiency}), xBar utilization (Figure~\ref{fig:xbar_utilization}), 
and robustness under noise (Figure~\ref{fig:noise_impact}). The chosen models span a range of parameter sizes, layer shapes, and computational footprints suitable for resource-constrained deployment.

\noindent
\textbf{Accuracy and Model Variants.} 
Figure~\ref{fig:accuracy_comparison} shows top-1 classification accuracy for each network under various baseline quantizations (FP16, Q8, Q4), along with “LeCA”~\cite{ma2023leca} from prior work~\cite{ma2023leca}, 
and our diode-based activation (DA) both with and without noise-aware (NA) training. 
In MobileNetV3-Small, the FP16 and Q8 baselines achieve 67.4\% and 64.9\% accuracy, while Q4 dips to 62.1\%. 
Our diode-based pipeline alone scores 59.82\%, but adding noise-aware training recovers performance to 61.77\%; an MCU-based refinement phase adds a slight boost to 61.98\%. 
Although these remain marginally below the vanilla Q4 baseline, the overall system power efficiency (Section~\ref{sec:SystemIntegration}) is significantly enhanced. 
For EfficientNet-Lite0, Q4 baseline accuracy is 71.9\%; our DA+NA approach reaches 71.02\%, and an MCU post-processing step yields 71.78\%. 
Similar trends appear with MicroNet~(M0) and MCUNet, where the diode activation alone degrades accuracy by about 2–3\%, but noise injection narrows this gap to under 1\%. 
Despite slight reductions relative to the pure Q4 software baseline, these hardware-oriented compromises retain acceptable accuracy within an ultra-low-power envelope. 
We also evaluate MobileViT-XXS, finding that DA+NA recovers most of the Q4 baseline accuracy (65.25\% vs.\ 66.2\%). 
The net result is that diode-based analog layers, combined with noise-aware training, can closely match the standard 4-bit baseline while reducing overhead from repeated ADC/DAC conversions.

\noindent
\textbf{Power Efficiency.} 
Figure~\ref{fig:power_efficiency} compares the achieved giga-operations per watt (GOps/W) across five different hardware platforms and our design. 
MobileNetV3-Small sees an efficiency of 196.70\,GOps/W with our approach, exceeding Nebula and HARDSEA by a modest margin (around 2--3\%) 
and surpassing ISAAC’s 152\,GOps/W by roughly 29\%. 
EfficientNet-Lite0 further highlights the energy advantages of our design, reaching 225.93\,GOps/W, about 50\% higher than ISAAC and 21\% above Nebula. 
Interestingly, MicroNet~(M0) yields 175.88\,GOps/W on our platform, which is lower than the 190\,GOps/W reported by HARDSEA; 
we attribute this to MicroNet’s comparatively small and irregular layer shapes that underutilize the two-layer analog pipeline, suggesting that dedicated tile-level scheduling can mitigate some inefficiencies. 
By contrast, MCUNet (240.2\,GOps/W) and MobileViT-XXS (228.9\,GOps/W) both exhibit stronger gains, indicating that deeper or more uniform model structures benefit more from our fused analog-layer approach.

\noindent
\textbf{xBar Utilization.} 
Figure~\ref{fig:xbar_utilization} reports average crossbar utilization across all network layers for each architecture. 
Our system maintains between 81\% and 96\% utilization, reflecting the tile-level scheduling that overlaps computation effectively. 
In MobileNetV3-Small, we reach 89.59\% utilization, surpassing ISAAC and Nebula by 7.2\% and 4.4\%, respectively. 
For MicroNet~(M0), we observe up to 96.18\% utilization, outperforming the nearest baseline (ResiRCA) by about 1.07\%. 
One exception is MobileViT-XXS, where HARDSEA slightly edges us out (87.14\% vs.\ 85.99\%). 
We find that certain skip connections and layer partitioning in MobileViT reduce concurrency opportunities, so exploring more fine-grained scheduling or dataflow tiling could further boost utilization.

\noindent
\textbf{Noise Robustness.}
Figure~\ref{fig:noise_impact} plots classification accuracy vs.\ SNR for four architectures (ResiRCA, Nebula, RedEye, and ours). 
At 10\,dB, our design attains 58.33\%, surpassing RedEye and Nebula by over 3\% and ResiRCA by more than 11\%. 
As SNR increases to 60\,dB, we approach 64.43\%, maintaining a consistent lead. 
This resilience stems from explicit noise injection during training, combined with analog activations that the network learns to handle. 
Although the gap narrows at higher SNR, the stability at low SNR highlights the advantage of modeling real analog variation in the training loop.

\noindent
\textbf{Discussion and Takeaways.}
Across these five DNNs, our analog pipeline yields both strong power efficiency and near-parity accuracy compared to a pure Q4 digital baseline. 
While certain model types exhibit a small accuracy penalty due to diode forward drop and IR-drop accumulation, 
careful integration of noise-aware training mitigates much of that loss. 
For networks with more uniform layer shapes (EfficientNet-Lite0, MCUNet), we see especially high efficiency gains, 
whereas specialized designs like HARDSEA can outperform us in smaller or irregular topologies like MicroNet~(M0). 
Overall, these results confirm that combining a diode-based analog datapath with hardware-aware training yields an effective strategy for deploying DNNs under stringent power budgets, 
while still preserving the accuracy necessary for many real-world edge tasks.


\begin{figure}[]
\centering
\includegraphics[width=\linewidth]{figs/samples.pdf}
\caption{Sample images from the NTLNP dataset.\vspace{-10pt}}
\label{fig:sample}
\end{figure}

\begin{figure*}[htbp]
\centering
\begin{subfigure}[t]{0.32\linewidth}
    \centering
    \includegraphics[width=\linewidth, height=1in]{NewResultFigs/power_efficiency_comparison_updated.pdf}
    \caption{Power efficiency comparison (GOps/W).}
    \label{fig:power_efficiency_case_study}
\end{subfigure}%
\hfill
\begin{subfigure}[t]{0.32\linewidth}
    \centering
    \includegraphics[width=\linewidth, height=1in]{NewResultFigs/crossbar_utilization_updated.pdf}
    \caption{Accuracy comparison.}
    \label{fig:accuracy_case_study}
\end{subfigure}%
\hfill
\begin{subfigure}[t]{0.32\linewidth}
    \centering
    \includegraphics[width=\linewidth, height=1in]{NewResultFigs/Accuracy_vs_noise_EH.pdf}
    \caption{Classification accuracy at different SNRs.}
    \label{fig:classification_accuracy_snr_case_Study}
\end{subfigure}
\vspace{-8pt}
\caption{Performance and accuracy evaluation on the NTLNP dataset under energy harvesting conditions.}
\label{fig:combined_case_study}
\end{figure*}


\subsection{A Case Study: Wildlife Monitoring under Energy Harvesting}
\label{sec:evalCaseStudy}
This case study highlights the advantages of our analog hardware design for wildlife monitoring in remote, energy-harvesting conditions. We use the Northeast Tiger and Leopard National Park (NTLNP) dataset~\cite{ntlnpdataset, tan2022animal, ntlnprepo}, which contains over 25,000 infrared-camera images of 17 different species (including Amur tigers, leopards, and bears) with variations in lighting, occlusions, and weather. Such challenging conditions amplify the need for ultra-low-power hardware that can sustain reliable inference despite intermittent power and analog noise.

\noindent\textbf{Importance of the Problem:} Deploying AI models for wildlife monitoring in remote locations requires hardware that is not only energy-efficient but also capable of operating under variable power conditions, such as those provided by energy harvesting sources (e.g., solar panels). Traditional digital hardware may consume too much power or lack the robustness needed for such environments. Our analog neural network hardware is well-suited for this application due to its ultra-low-power consumption, high computational efficiency, and resilience to hardware non-idealities, making it ideal for deployment in energy-harvesting systems.

\noindent\textbf{Models and Implementation:} We fine-tuned MobileNetV3-Small, EfficientNet-Lite0, MicroNet (M0) and MCUNet, on the NTLNP dataset, to perform image classification tasks. These models vary in complexity, allowing us to assess the performance of our hardware across different computational demands. The models were trained using our hardware-aware training process, incorporating diode-based activation functions and IR drop modeling to enhance robustness and accuracy when deployed on analog hardware.

\noindent\textbf{Energy-Harvesting Scenario:} The computation was powered by solar energy traces obtained from the NOAA SOLRAD database~\cite{solrad}. This setup simulates a realistic energy harvesting environment, where the available power fluctuates throughout the day as a result of changes in solar irradiance. 

\noindent\textbf{Hardware Changes:} To cater towards the variable power budget, we include a power predictor~\cite{qiu2020resirca, usas} that indicates the control block to augment power-aware scheduling~\cite{usas}. To further enhance computation efficiency, a loop-tiling based computation decomposition technique~\cite{qiu2020resirca} was used to maximize the forward progress with minimum energy.

\noindent\textbf{Results Discussion: }Figure~\ref{fig:power_efficiency_case_study} compares power efficiency (GOps/W) against ResiRCA and RedEye. Our design consistently outperforms the baselines, particularly in variable solar conditions that degrade efficiency in conventional platforms. For MCUNet, we achieve 228.56\,GOps/W versus 136\,GOps/W on ResiRCA and 38\,GOps/W on RedEye; this near twofold improvement stems from our analog pipeline’s reduced ADC/DAC overhead and robust scheduling under fluctuating power. Similar gains appear in EfficientNet-Lite0, where we reach 225.93\,GOps/W compared to ResiRCA’s 134.87\,GOps/W. MobileNetV3-Small operates at 166.70\,GOps/W, also exceeding ResiRCA under these constrained conditions.

Figure~\ref{fig:accuracy_case_study} shows final classification accuracy on the NTLNP dataset. Our 4-bit implementation surpasses both the 8-bit and 4-bit variants of ResiRCA, as well as RedEye. For MobileNetV3-Small, we obtain 95.8\% accuracy compared to 90.6\% in ResiRCA (Q4) and 88.75\% in RedEye; EfficientNet-Lite0 improves to 88.1\% versus 83.2\% (Q4 ResiRCA) and 82.04\% (RedEye). Despite operating under low precision and fluctuating voltage supplies, the network adapts to analog imperfections through diode-aware and noise-aware training. We observe similar trends in MicroNet (82.06\% vs.\ 77.56\%) and MCUNet (76.1\% vs.\ 72.2\%), indicating that even smaller-edge architectures benefit from our method.

Figure~\ref{fig:classification_accuracy_snr_case_Study} plots accuracy versus SNR for two representative networks, EfficientNet-Lite0 and MCUNet. At 10\,dB, the former retains 70.40\% accuracy, rising to 86.80\% by 60\,dB. MCUNet starts at 51.80\% accuracy under heavy noise (10\,dB) but converges to 74.10\% at 60\,dB. Such resilience under low SNR highlights the importance of explicitly modeling analog noise and IR drops in the training loop. Baseline methods, which do not incorporate hardware-aware noise injection or diode-level variations, experience more severe degradation when voltage fluctuations and intermittent power exacerbate analog errors.
Compared to traditional benchmarking under fixed power, these energy-harvesting experiments reveal the unique advantages of hardware-aware analog computing. By designing for intermittent power from the outset, our system avoids the steep performance losses that conventional digital or partially analog architectures face under dynamic supply constraints. The reduced conversion overhead, combined with analog resilience to noise, enables steady performance across diurnal cycles. This makes our approach well-suited for continuous field monitoring of wildlife habitats, where reliability and energy autonomy are paramount.

\begin{table}[]
\centering
\begin{tabular}{lcccc}
\hline
& \multicolumn{2}{c}{\textbf{DistilBERT-Tiny}} & \multicolumn{2}{c}{\textbf{ViT-Micro}} \\
\textbf{Precision} & ReLU & Diode & ReLU & Diode \\
\hline
\textbf{FP32}  & 93.2 & 93.0 & 72.1 & 71.9 \\
\textbf{FP16}  & 93.0 & 92.8 & 71.9 & 71.7 \\
\textbf{INT16} & 92.7 & 92.5 & 71.3 & 71.2 \\
\textbf{INT8}  & 92.1 & 91.8 & 70.6 & 70.3 \\
\hline
\end{tabular}
\caption{{Accuracy (\%) on two small transformer models---DistilBERT-Tiny (GLUE) and ViT-Micro (Tiny-ImageNet)
%---under different quantizations.
}
\vspace{-16pt}
\label{tab:transformerAcc1}
}
\vspace{-8pt}
\end{table}

\subsection{Extending to Small Transformer Models}
\label{sec:transformerEval}
{
Motivated by the recent PIM based works on transformers~\cite{wolters2024memory, guo2024towards, yang2020retransformer}, we evaluated our analog activation for small-scale transformers in natural language and vision tasks. We substituted ReLU with our diode-based function in \textit{DistilBERT-Tiny}~\cite{jiao2019tinybert}, a 4-layer language model evaluated on the GLUE benchmark~\cite{wang2018glue} (with a hidden dimension of 256), and in \textit{ViT-Micro}~\cite{setyawan2025microvit}, a 4-block vision transformer on Tiny-ImageNet~\cite{le2015tiny} (with 6 attention heads of dimension 192).}

{
After replacing the ReLU activations, we fine-tuned each transformer under various quantization levels, including floating-point 32-bit (\textbf{FP32}), floating-point 16-bit (\textbf{FP16}), and integer fixed-point precision at 16-bit (\textbf{INT16}) and 8-bit (\textbf{INT8}). Table~\ref{tab:transformerAcc1} summarizes the test accuracies before and after inserting our diode-based activation. Overall, the accuracy drop compared to ReLU is minimal: in both DistilBERT-Tiny and ViT-Micro, diode-based activations achieve results within $0.5\%$ of ReLU even at INT8 precision. Moreover, swapping ReLU for diode activation did not necessitate major architecture changes or specialized hyperparameters; both models adapted effectively over 5--10 additional fine-tuning epochs.
}

{
These findings suggest that the proposed analog diode-based activation can extend beyond CNNs to small transformer models for both language and vision tasks, preserving accuracy under moderate or strict quantization. Our future work aims to investigate deeper transformer configurations (12--24 layers) to further validate our robustness of our proposed method for advanced transformers.
}
%%%%% %%%%% %%%%%


% %-------------------------------------------------------------------------------

% %-------------------------------------------------------------------------------
% %\vspace{-4pt}
% \section{Related Work}
% \label{sec:relwork}
% \section{Related Work} 
% Processing-in-memory (PIM) architectures have gained prominence for accelerating deep neural networks, especially using analog or resistive memory devices to perform in-situ computation. Early analog PIM designs like ISAAC \cite{Shafiee16} and PRIME \cite{Chi16} demonstrated the feasibility of using ReRAM crossbar arrays to execute neural network inference by performing analog matrix-vector multiplications directly in memory. ISAAC introduced an accelerator that stores weights in memristive crossbars and executes CNN layers with in-situ analog dot products, relying on high-precision ADCs after each crossbar to handle multi-bit outputs. PRIME explored embedding neural computation within ReRAM-based main memory, showing that a portion of memory arrays could be repurposed for neural computing to reduce data movement. These and other mixed-signal accelerators confirmed that analog crossbars can dramatically improve energy efficiency for DNN inference. Another notable direction is RedEye \cite{LiKamWa16}, an image sensor architecture that pushes early convolutional processing into the analog domain. RedEye performs initial ConvNet layers analogously at the sensor focal plane before quantization, thereby alleviating the costly analog-to-digital conversion at the imager output. All of the above efforts, however, constrain the analog computation to essentially one layer at a time (one crossbar operation followed by digitization). Traditional non-linear activations (e.g., ReLU) are implemented outside the crossbar in the digital domain, which forces intermediate analog results to be converted to digital between layers. This limitation means prior analog PIM systems still incurred per-layer conversion overheads and could not truly cascade multiple neural layers entirely in analog. Some works have looked at extending PIM to training or deeper pipelines (e.g., PipeLayer \cite{Song17HPCA} supports weight updates in ReRAM and a pipelined flow), but they similarly stop short of enabling consecutive analog layers with native analog non-linearities. 

% In tandem with architectural advances, researchers have developed hardware-aware training techniques to cope with analog non-idealities. Deep networks mapped to analog substrates can suffer accuracy loss from device variability, IR drop across crossbar interconnects, sense amplifier noise, quantization effects, and other analog imperfections. To mitigate this, prior works incorporate circuit effects into the training process so that the model learns to compensate for hardware errors. For example, Song et al. propose an “imperfection-tolerable” training algorithm for ReRAM-based accelerators that models device non-idealities (e.g., conductance variations, IR drop) during backpropagation to preserve accuracy on crossbar hardware \cite{Song21TCAD}. More broadly, several studies inject noise and quantization errors into training or retraining phases to immunize the network against analog variability at inference time. Such co-design approaches significantly improve the robustness of analog accelerators, but earlier implementations still assumed the conventional layer-by-layer processing structure with analog computing limited to the MAC operations. In contrast to these efforts, our work tightly couples model and hardware design to enable a deeper analog execution pipeline. We introduce a Schottky diode-based analog activation function at the crossbar outputs, allowing two consecutive layers to be computed entirely in the analog domain without intermediate digitization. This innovation builds on the insights of hardware-aware learning: during training, we explicitly model crossbar IR drops, diode I–V characteristics shifts, amplifier noise, and quantization in the loop so that the network adapts to the analog hardware’s imperfections. By co-designing the DNN (architecture and training procedure) with the circuit primitives, our approach achieves what prior PIM systems could not: an analog accelerator that evaluates multiple neural layers back-to-back analogly, significantly reducing costly A/D conversions while maintaining accuracy. This combination of a novel analog activation mechanism and integrated analog-aware training distinguishes our work from prior PIM solutions in both architecture and methodology.

\section{Related Work}

Processing-in-memory (PIM) architectures leveraging analog or resistive memories have demonstrated substantial energy savings for DNN inference by performing matrix–vector multiplications in situ. Pioneering designs such as ISAAC~\cite{shafiee2016isaac} and PRIME~\cite{chi2016prime} store weights in ReRAM crossbars and rely on ADCs after each array to digitize partial sums, while RedEye~\cite{RedEye2020} embeds early convolutional layers into the image sensor to reduce conversion overhead. These systems, however, execute only one analog layer per crossbar cycle and implement non‑linear activations (e.g., ReLU) in the digital domain, incurring frequent A/D–D/A conversions. Extensions like PipeLayer~\cite{song2017pipelayer} support training and pipelined flows but still cannot cascade multiple analog layers with native non‑linearities.

Complementing architectural advances, hardware‑aware training methods inject analog non‑idealities—device variability, IR drop, noise, and quantization—into the training loop so that networks learn to compensate for circuit errors~\cite{xu2022multi}. While such co‑design improves robustness, prior work assumes a layer‑by‑layer pipeline with analog MACs only. In contrast, we propose a Schottky diode–based analog activation at each crossbar output and a two‑layer analog compute pipeline, eliminating intermediate digitization. By modeling crossbar IR drop, diode I–V shifts, amplifier errors, and quantization during training, our approach tightly fuses model and hardware design, enabling back‑to‑back analog execution and significantly reducing conversion overheads without sacrificing accuracy.

% %-------------------------------------------------------------------------------

% %-------------------------------------------------------------------------------
%\vspace{-4pt}
% \section{Conclusions}
% \label{sec:conclusion}
% % \section{Conclusion}
% % \label{sec:conclusion}

% In this paper, we have presented a holistic hardware-model co-design that significantly enhances the efficiency and robustness of ReRAM-based PIM architectures for DNN inference in ultra-low-power, energy-harvesting environments. By integrating analog-aware training techniques and implementing activation functions directly in the analog domain using Schottky diodes, we addressed the limitations of traditional architectures that suffer from energy and latency overheads due to frequent A2D and D2A conversions and are restricted to computing only one layer in the analog domain.

% Our analog-aware training methodology models analog circuit behaviors—including IR drop, device variability, and thermal noise—into the DNN training process. This enables the neural network to learn compensatory mechanisms that mitigate hardware-induced errors, maintaining high classification accuracy despite analog non-idealities. The implementation of diode-based activation functions allows computations for two consecutive neural network layers to be performed entirely in the analog domain without intermediate conversions, thus reducing energy consumption and latency.

% The experimental results demonstrate that our proposed design achieves up to 68\% higher power efficiency compared to the baseline platforms. On benchmark datasets, our analog-aware trained models maintained high classification accuracy, achieving up to 1.63\% improvement over 4-bit quantized models. In the case study using the NTLNP~\cite{ntlnpdataset, ntlnprepo} wildlife image dataset, our system achieved an accuracy of 88.1\% with MicroNet~\cite{li2021micronet}, outperforming the 4-bit baseline by 4.9\% and RedEye~\cite{RedEye2020} by 6.06\%. These results validate the effectiveness of our approach in real-world energy-constrained applications.

% By bridging the gap between theoretical models and hardware implementations, our work enables efficient and reliable DNN deployment in environments with stringent energy constraints, paving the way for advanced edge computing applications like wildlife conservation, remote sensing, and industrial automation.
% %Reducing the dependency on A2D and D2A conversions and enhancing robustness to analog non-idealities paves the way for advanced edge computing applications in fields like wildlife conservation, remote sensing, and industrial automation.

\section{Conclusions}
\label{sec:conclusions}
In this paper, we have introduced a hardware-software co-design that tackles the inherent challenges of ReRAM-based PIM for ultra-low-power, energy-harvesting edge deployments. By integrating a diode-based activation circuit with IR-drop and noise-aware training, our approach enables two consecutive DNN layers to be executed entirely in the analog domain, thus reducing repeated ADC/DAC overhead and enhancing power efficiency. The proposed methodology incorporates realistic device and circuit variations, ensuring that the network learns to compensate for IR drops, diode forward voltage, and thermal or flicker noise.

Experimental results across a range of popular DNNs, including MobileNetV3-Small, EfficientNet-Lite0, and MCUNet, show that our design achieves significant improvements in power efficiency—often 20–50\% higher (or more) than state-of-the-art analog accelerators like ISAAC~\cite{shafiee2016isaac}, ResiRCA~\cite{qiu2020resirca}, and Nebula~\cite{singh2020nebula}—while retaining competitive accuracy under 4-bit quantization. In a comprehensive wildlife monitoring case study on the NTLNP dataset~\cite{ntlnpdataset, ntlnprepo}, our system reaches 95.8\% accuracy with MobileNetV3-Small and 88.1\% with EfficientNet-Lite0 under variable solar power conditions, outperforming baseline 8-bit and 4-bit platforms. These gains reflect how hardware-aware training, grounded in circuit-level modeling of diode activations and crossbar non-idealities, enables robust, high-efficiency inference in intermittent power scenarios.

Overall, this work demonstrates that bridging analog circuit physics and deep learning models can unlock notable energy and latency savings without compromising performance. By addressing ADC/DAC overheads, device variability, and noise through a unified training framework, we pave the way for resource-efficient DNN deployment in remote or battery-operated environments such as wildlife monitoring, environmental sensing, and industrial IoT.


% %-------------------------------------------------------------------------------

%%%%%%% -- PAPER CONTENT ENDS -- %%%%%%%%
%%%%%%%%%%%%%%%%%%%%%%%%%%%%%%%%%%%%%%%%%%%%%%%%%%%

%%%%%%%%%%%%%%%% CONTENT ENDS HERE %%%%%%%%%%%%%%%%

%%%%% %%%% %%%%%%%% %%%%%
%%
%% The next two lines define the bibliography style to be used, and
%% the bibliography file.
\bibliographystyle{ACM-Reference-Format}
\bibliography{sample-base}


%%
%% If your work has an appendix, this is the place to put it.

\end{document}
\endinput
%%
%% End of file `sample-sigconf.tex'.
