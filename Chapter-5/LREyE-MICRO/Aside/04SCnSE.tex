Mathematical Modeling of IR Drop in ReRAM Crossbar Arrays with Variable Memristor Resistance


ReRAM crossbar arrays offer a promising architecture for dense memory storage and in-memory computing applications, particularly in Deep Neural Networks (DNNs). However, scaling these arrays introduces significant challenges related to voltage drops across the word and bit lines, which can degrade signal integrity and system performance. This paper aims to provide a detailed mathematical model for IR drop in ReRAM crossbars, accounting for variable memristor resistances, and to establish guidelines for determining optimal array sizes.



\textbf{Crossbar Structure}
A ReRAM crossbar array consists of \( m \) word lines (rows) and \( n \) bit lines (columns) intersecting at \( m \times n \) crosspoints. Each crosspoint houses a memristor that stores data by varying its resistance state. In a 4-bit per memristor configuration, each memristor can represent \( 2^4 = 16 \) distinct resistance levels.

\textbf{Memristor Specifications}
\begin{itemize}
    \item \textbf{4-Bit Representation:} Each memristor encodes 4 bits of information through 16 distinct resistance states.
    \item \textbf{Resistance States:} Let \( R_{\text{min}} \) and \( R_{\text{max}} \) denote the minimum and maximum resistances, respectively. The resistance levels are defined as:
    \[
    R_{\text{mem}} \in \{ R_0, R_1, R_2, \dots, R_{15} \}
    \]
    where \( R_0 = R_{\text{min}} \) and \( R_{15} = R_{\text{max}} \).
    \item \textbf{Programming:} Each memristor is programmed to a specific resistance state based on the data it needs to store.
\end{itemize}

\subsection{Mathematical Modeling of IR Drop}

\textbf{Basic Assumptions}
\begin{enumerate}
    \item \textbf{Ohmic Lines:} Word and bit lines are treated as resistive conductors with uniform resistance per unit length.
    \item \textbf{Steady-State Conditions:} The analysis is conducted under steady-state DC conditions.
    \item \textbf{Uniform Current Distribution:} Current is uniformly distributed unless influenced by IR drops.
    \item \textbf{Stable Memristor States:} Memristor resistances are assumed to remain constant during read/write operations.
\end{enumerate}

\textbf{Resistive Network Parameters}
\begin{itemize}
    \item \textbf{Word Line Resistance (\( R_W \)):} Resistance per word line.
    \item \textbf{Bit Line Resistance (\( R_B \)):} Resistance per bit line.
    \item \textbf{Memristor Resistance (\( R_{\text{mem}_{ij}} \)):} Variable resistance at each crosspoint \( (i, j) \) based on its programmed state.
\end{itemize}

\textbf{Voltage Drop Analysis}

To model the IR drop, we calculate the voltage at each crosspoint by considering the resistances of the word and bit lines along with the variable memristor resistances.

\textbf{Single-Row or Single-Column Analysis}

For simplification, analyze a single row or column:
\begin{align*}
    R_W^{\text{total}} &= R_W \times (n + 1) \quad \text{(assuming \( n \) memristors per word line)} \\
    R_B^{\text{total}} &= R_B \times (m + 1) \quad \text{(assuming \( m \) memristors per bit line)}
\end{align*}

\textbf{General \( m \times n \) Array Analysis}

Using Kirchhoff's Current Law (KCL) and Kirchhoff's Voltage Law (KVL), the voltage at each node in the crossbar can be determined by solving a system of linear equations.

\paragraph{Matrix Representation}
Define a conductance matrix \( G \) where each element \( G_{ij} \) represents the conductance between nodes \( i \) and \( j \). The system is represented as:
\[
G \cdot V = I
\]
where:
\begin{itemize}
    \item \( V \) is the vector of node voltages.
    \item \( I \) is the vector of input currents.
    \item \( G \) incorporates the conductances of word lines, bit lines, and memristors.
\end{itemize}

\paragraph{Simplified Approach}
For large arrays, exact solutions are computationally intensive. Approximate models or iterative solvers such as Gauss-Seidel or Conjugate Gradient methods are employed for practical analysis.

\textbf{Analytical Model for Voltage at a Crosspoint}

Assuming symmetric resistive properties and uniform memristor states, the voltage at a crosspoint \( (i, j) \) can be approximated by:
\[
V_{ij} = V_{\text{in}} \left( \frac{R_B}{R_W + R_B} \right)
\]
This simplification assumes an evenly distributed load, which is rarely the case in practical scenarios.

\textbf{Incorporating Variable Memristor Resistance}

To enhance the model's accuracy, variable memristor resistances must be integrated into the resistive network.

\textbf{Modified Conductance Matrix}
Each memristor's conductance \( G_{\text{mem}_{ij}} = \frac{1}{R_{\text{mem}_{ij}}} \) varies based on its resistance state. The conductance matrix \( G \) thus becomes a function of the individual \( G_{\text{mem}_{ij}} \) values:
\[
G \cdot V = I
\]
where \( G \) now incorporates variable conductances:
\[
G = \begin{bmatrix}
G_{\text{WL}_1} & G_{\text{mem}_{11}} & \dots & G_{\text{mem}_{1n}} \\
G_{\text{mem}_{21}} & G_{\text{WL}_2} & \dots & G_{\text{mem}_{2n}} \\
\vdots & \vdots & \ddots & \vdots \\
G_{\text{mem}_{m1}} & G_{\text{mem}_{m2}} & \dots & G_{\text{WL}_m}
\end{bmatrix}
\]
where \( G_{\text{WL}_i} = \frac{1}{R_W} \) and \( G_{\text{mem}_{ij}} = \frac{1}{R_{\text{mem}_{ij}}} \).

\textbf{System of Equations}
For each node \( (i, j) \), KCL is applied:
\[
\sum_{k} G_{ik} (V_i - V_k) = I_i
\]
This results in a system of \( m \times n \) linear equations that can be solved numerically.

\textbf{Voltage Drop Constraints for Signal Integrity}

Signal integrity is maintained if the voltage drop \( \Delta V \) across the lines does not exceed a certain threshold relative to the input voltage \( V_{\text{in}} \). This can be defined as:
\[
\Delta V = V_{\text{in}} - V_{ij} \leq \alpha \cdot V_{\text{in}}
\]
where \( \alpha \) is the maximum acceptable fraction of voltage drop (e.g., 5\%).

\textbf{Determining Maximum \( m \times n \) Size}

To ensure \( \Delta V \leq \alpha V_{\text{in}} \), derive a relationship between \( m \), \( n \), \( R_W \), and \( R_B \).

\textbf{Initial Approximation (Ignoring Variable \( R_{\text{mem}} \))}
\[
\Delta V = V_{\text{in}} \left( \frac{n \cdot R_W}{n \cdot R_W + m \cdot R_B} \right) \leq \alpha \cdot V_{\text{in}}
\]
Simplifying:
\[
\frac{n \cdot R_W}{n \cdot R_W + m \cdot R_B} \leq \alpha \\
\Rightarrow n \cdot R_W \leq \alpha (n \cdot R_W + m \cdot R_B) \\
\Rightarrow n \leq m \cdot \frac{\alpha R_B}{(1 - \alpha) R_W}
\]
\[
\frac{n}{m} \leq \frac{\alpha R_B}{(1 - \alpha) R_W} \\
\Rightarrow n \leq m \cdot \frac{\alpha R_B}{(1 - \alpha) R_W}
\]

\textbf{Revised Constraint with Variable \( R_{\text{mem}} \)}

Considering the average memristor resistance \( R_{\text{mem}_{\text{avg}}} \):
\[
n \leq m \cdot \frac{\alpha R_B}{(1 - \alpha) R_W + \alpha R_{\text{mem}_{\text{avg}}}}
\]

% \textbf{Example Calculation}

% Given:
% \begin{itemize}
%     \item \( V_{\text{in}} = 1\,\text{V} \)
%     \item \( R_W = 10\,\Omega \)
%     \item \( R_B = 10\,\Omega \)
%     \item \( \alpha = 0.05 \) (5\%)
%     \item \( R_{\text{mem}_{\text{avg}}} = 50\,\Omega \)
% \end{itemize}

% Calculate:
% \[
% n \leq m \cdot \frac{0.05 \times 10}{(1 - 0.05) \times 10 + 0.05 \times 50} = m \cdot \frac{0.5}{9.5 + 2.5} = m \cdot \frac{0.5}{12} \approx m \cdot 0.0417
% \]
% Thus, for \( m = 100 \):
% \[
% n \leq 100 \times 0.0417 \approx 4.17
% \]
% Therefore, a \( 100 \times 4 \) crossbar maintains the voltage drop within 5\%, considering variable memristor resistances.

\subsection{Detailed Mathematical Modeling}

\textbf{Resistive Network Definition}

Each memristor connects a word line (WL) and a bit line (BL). The resistive network comprises:
\begin{itemize}
    \item \textbf{Word Lines:} \( m \) parallel word lines, each with resistance \( R_W \).
    \item \textbf{Bit Lines:} \( n \) parallel bit lines, each with resistance \( R_B \).
    \item \textbf{Memristors:} \( m \times n \) memristors, each with resistance \( R_{\text{mem}_{ij}} \).
\end{itemize}

\textbf{Kirchhoff’s Laws Application}

For each node \( (i, j) \), apply KCL:
\[
\sum_{k} \frac{V_{ij} - V_k}{R_{ik}} = I_{ij}
\]
where:
\begin{itemize}
    \item \( V_{ij} \) is the voltage at node \( (i, j) \).
    \item \( V_k \) are the voltages at connected nodes.
    \item \( R_{ik} \) are the resistances between node \( (i, j) \) and node \( k \).
\end{itemize}

\textbf{Conductance Matrix Formulation}

Construct the conductance matrix \( G \) where:
\[
G_{ij} = \sum \text{Conductances connected to node } i
\]
\[
G_{ij} = \begin{cases}
    \sum \text{Conductances connected to node } i & \text{if } i = j \\
    -\text{Conductance between } i \text{ and } j & \text{if } i \neq j
\end{cases}
\]

\textbf{Solving the System}

The system \( G \cdot V = I \) can be solved using numerical methods:
\[
V = G^{-1} \cdot I
\]
For large \( m \times n \) arrays, iterative solvers such as Gauss-Seidel or Conjugate Gradient methods are preferred due to computational efficiency.

\textbf{Calculating IR Drop}

Once \( V \) is determined, the IR drop across word and bit lines is:
\[
\Delta V_{\text{WL}} = V_{\text{in}} - V_{\text{WL}} \\
\Delta V_{\text{BL}} = V_{\text{BL}} - V_{\text{out}}
\]
where \( V_{\text{WL}} \) and \( V_{\text{BL}} \) are the voltages along the word and bit lines, respectively.

\subsection{Determining Maximum \( m \times n \) Size}

To ensure signal integrity:
\begin{itemize}
    \item \textbf{Define Voltage Drop Limit (\( \alpha \)):} Maximum allowable fraction of \( V_{\text{in}} \).
    \item \textbf{Apply Constraints:} Use the derived inequalities to relate \( m \), \( n \), \( R_W \), \( R_B \), and \( R_{\text{mem}_{\text{avg}}} \).
\end{itemize}

\textbf{Iterative Design Approach}
\begin{enumerate}
    \item \textbf{Start with a Small Array:} Choose initial \( m \times n \).
    \item \textbf{Simulate IR Drop:} Calculate \( \Delta V \) using the mathematical model.
    \item \textbf{Check Constraints:} Ensure \( \Delta V \leq \alpha V_{\text{in}} \).
    \item \textbf{Scale Up:} Incrementally increase \( m \) and \( n \) while monitoring \( \Delta V \).
    \item \textbf{Determine Maximum Size:} The largest \( m \times n \) where \( \Delta V \leq \alpha V_{\text{in}} \) defines the optimal array size.
\end{enumerate}

% \subsection{Additional System Considerations}

% \textbf{Memristor Variability and Reliability}
% \begin{itemize}
%     \item \textbf{Device Variability:} Fabrication inconsistencies can lead to variations in \( R_{\text{mem}} \), exacerbating IR drops.
%     \item \textbf{Endurance and Retention:} Repeated read/write cycles may alter memristor states, affecting resistances and IR drops over time.
% \end{itemize}

% \textbf{Thermal Effects}
% \begin{itemize}
%     \item \textbf{Self-Heating:} High current densities can cause localized heating, altering line resistances and memristor states.
%     \item \textbf{Thermal Management:} Implement effective cooling strategies to maintain stable operation.
% \end{itemize}

% \textbf{Crosstalk and Noise}
% \begin{itemize}
%     \item \textbf{Signal Interference:} Adjacent lines can interfere, especially in densely packed arrays.
%     \item \textbf{Noise Margins:} Ensure that signal levels remain distinguishable despite noise.
% \end{itemize}

% \textbf{Power Consumption}
% \begin{itemize}
%     \item \textbf{Dynamic Power:} Active switching of memristors consumes power, influencing thermal and IR drop considerations.
%     \item \textbf{Static Power:} Leakage currents contribute to overall power consumption.
% \end{itemize}

% \textbf{Integration with Peripheral Circuits}
% \begin{itemize}
%     \item \textbf{ADC/DAC Requirements:} Analog-to-digital and digital-to-analog converters are needed for interfacing, and their precision affects overall system accuracy.
%     \item \textbf{Sense Amplifiers:} Must account for IR drops to maintain signal integrity.
% \end{itemize}

% \textbf{Scalability and Fabrication Constraints}
% \begin{itemize}
%     \item \textbf{Manufacturing Limits:} Line widths, spacing, and memristor placement are constrained by fabrication technology.
%     \item \textbf{Yield and Defect Rates:} Larger \( m \times n \) sizes may lead to increased defect rates, impacting overall yield.
% \end{itemize}

% \textbf{Error Correction and Mitigation Techniques}
% \begin{itemize}
%     \item \textbf{Redundancy:} Incorporate redundant memristors or lines to compensate for faulty components.
%     \item \textbf{Calibration:} Periodic calibration can adjust for drift and variability in resistance states.
% \end{itemize}

% \textbf{Circuit Design Optimization}
% \begin{itemize}
%     \item \textbf{Material Selection:} Use low-resistivity materials for word and bit lines to minimize IR drops.
%     \item \textbf{Topology Optimization:} Design the layout to minimize the length and maximize the width of critical lines.
% \end{itemize}

% \textbf{Algorithmic Considerations}
% \begin{itemize}
%     \item \textbf{Quantization:} Employ algorithms robust to variations in resistance and voltage levels.
%     \item \textbf{Error-Tolerant Computations:} Design DNN architectures that tolerate slight inaccuracies due to IR drops.
% \end{itemize}

% \subsection{Recommendations for Enhanced Modeling and Simulation}

% \textbf{Utilize Numerical Simulation Tools}
% Employ tools like SPICE, MATLAB, or specialized circuit simulators to numerically solve the conductance matrix \( G \cdot V = I \) with actual \( R_{\text{mem}} \) distributions.

% \textbf{Incorporate Statistical Variability}
% Model the distribution of memristor resistances based on fabrication data or anticipated programming patterns using statistical methods to estimate average and worst-case scenarios for \( \Delta V \).

% \textbf{Implement Iterative Solvers}
% For large \( m \times n \) arrays, use iterative solvers (e.g., Gauss-Seidel, Conjugate Gradient) to efficiently compute node voltages without explicitly inverting large matrices.

% \textbf{Explore Memristor Programming Strategies}
% Optimize programming algorithms to minimize high-resistance states where possible, reducing the average \( R_{\text{mem}} \) and mitigating IR drops.

% \textbf{Design for Redundancy and Error Correction}
% Introduce redundant rows/columns or error correction mechanisms to compensate for areas with excessive IR drops or faulty memristors.

% \textbf{Optimize Crossbar Layout}
% Adjust the physical layout to minimize the length and resistance of critical word and bit lines. Techniques like segmentation or hierarchical interconnects can help manage IR drops more effectively.

% \textbf{Dynamic Voltage Compensation}
% Implement circuits that dynamically adjust the input voltage or employ voltage droop compensation techniques to maintain consistent voltage levels across the crossbar despite variable memristor resistances.


Incorporating variable memristor resistance into the IR drop model of ReRAM crossbar arrays significantly enhances the accuracy and reliability of the analysis. This comprehensive approach is essential for designing scalable and robust ReRAM crossbar arrays capable of supporting complex Deep Neural Network (DNN) applications. Future work should focus on empirical validation of the models and the development of advanced mitigation strategies to handle variability-induced challenges.